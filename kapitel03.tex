\chapter{Dokumentklassen und Seitenstile}
\label{KapitelDokumentklassenSeitenstile}

Die in diesem Kapitel beschriebenen Dokumentklassen und Seitenstile definieren die grundlegenden Parameter des Layouts eines Dokuments. Zudem präsentiert dieses Kapitel die benötigten Befehle und Umgebungen, um Dokumente sinnvoll zu gliedern. 

\section{Dokumentklassen}
\label{AbschnittDokumentklassen}
\index{Dokumentklasse}

Der erste Befehl, der in der Präambel einer \LaTeX-Quelldatei (\verb!.tex!-Datei) steht, ist der Befehl \verb!\documentclass!
\index[cmd]{\texttt{\textbackslash documentclass}}. Dieser definiert im Parameter \verb!Dokumentklasse!
die Dokumentklasse für das gesamte Dokument. 

\fbox{\texttt{\textbackslash documentclass[}\textsl{Optionen}\texttt{]\{}\textsl{Dokumentklasse}\texttt{\}}}

Mehr als eine Dokumentklasse pro Dokument ist unzulässig. Moderne \LaTeX-Distributionen bringen eine große Anzahl an Dokumentklassen mit. Weil dieses Dokument aus Platzgründen nicht alle beschreiben kann, liegt an dieser Stelle der Fokus auf denjenigen Dokumentklassen, die seit Jahrzehnten sehr populär sind, und die in der Literatur und in der Praxis häufig eingesetzt werden. Zusätzlich berücksichtigt diese Dokumentation einige der neueren KOMA-Klassen~\cite{KOMAScript_Dokumentation}, die sich speziell den Gepflogenheiten des deutschen Sprachraums widmen. 

Im Detail handelt es sich bei den hier vorgestellten traditionellen Dokumentklassen um 
\verb!article!\index[cmd]{\texttt{article}}, 
\verb!book!\index[cmd]{\texttt{book}}, 
\verb!proc!\index[cmd]{\texttt{proc}}, 
\verb!report!\index[cmd]{\texttt{report}}, 
\verb!letter!\index[cmd]{\texttt{letter}}
und \verb!slides!\index[cmd]{\texttt{slides}}. 



Die Dokumentklasse 
\verb!article! eignet sich gut für kleinere bis mittelgroße Dokumente. Texte, die mit dieser Dokumentklasse realisiert werden, können in
Abschnitte (\textsl{Sections}), Unterabschnitte (\textsl{Subsections}) usw. untergliedert werden. Kapitel (\textsl{Chapter}) kennt diese Klasse aber leider nicht.

Für längere Texte, die in Kapitel, Abschnitte, Unterabschnitte usw. unterteilen werden sollen, eignet sich u.a. die Dokumentklasse \verb!report!.
Für Bücher oder sonstige umfangreiche Dokumente, die in Kapitel,
Abschnitte und Unterabschnitte untergliedert sein sollen, eignet sich die Dokumentklasse \verb!book!. Bei den Dokumentklassen \verb!book! und \verb!report! beginnen Kapitel auch immer mit einer neuen Seite.

Speziell zur Erstellung von Sitzungsprotokollen eignet sich die Dokumentklasse \verb!proc!.

Die Dokumentklasse \verb!letter! ist eine auf US-amerikanische 
Briefe zugeschnittene Klasse. Für Briefe nach deutschem Standard sind 
andere Klassen wie \verb!dinbrief!\index[cmd]{\texttt{dinbrief}},
\verb!g-brief! oder das modernere \verb!g-brief2!\index[cmd]{\texttt{g-brief2}} besser geeignet (siehe Kapitel~\ref{Kapitel_Briefe}). 


Die Dokumentklasse \verb!slides! eignet sich für Präsentationsfolien. Eine modernere Alternative ist \verb!beamer!\index[cmd]{\texttt{beamer}} (siehe Kapitel~\ref{Kapitel_Praesentationsfolien}).


Bei den moderneren KOMA-Dokumentklassen handelt es sich 
um \verb!scrartcl!\index[cmd]{\texttt{scrartcl}} (als Ersatz für \verb!article!),  
\verb!scrreprt!\index[cmd]{\texttt{scrreprt}} (als Ersatz für \verb!report!), 
\verb!scrbook!\index[cmd]{\texttt{scrbook}} (als Ersatz für \verb!book!) und 
\verb!scrlttr2!\index[cmd]{\texttt{scrlttr2}} (als Ersatz für \verb!letter!).






\subsection{Klassenoptionen}
\index{Klassenoptionen}

Der Befehl \verb!\documentclass! bietet die 
Möglichkeit, Optionen in 
eckigen Klammern anzugeben. 
Mit diesen wird das grundlegende Layout eines Dokuments 
angepasst. 

In den Optionen ist es unter anderem möglich, die Schriftgröße\index{Schriftgröße} der Basisschrift\index{Basisschrift} anzupassen, die Anzahl der Textspalten pro Seite und die Papiergröße sowie die Ausrichtung zu definieren.

Von den in diesem Abschnitt beschrieben Optionen muss keine zwingend angegeben werden. Sie sind optional. Sind mehrere Optionen angegeben, müssen diese mit Kommas voneinander getrennt sein, wie im folgenden Beispiel zu sehen ist:

\begin{Verbatim}[frame=single]
\documentclass[12pt,twoside,a4paper]{report}
\end{Verbatim}

\subsubsection{\texttt{10pt}, \texttt{11pt} oder \texttt{12pt}}
\index[cmd]{\texttt{10pt}}
\index[cmd]{\texttt{11pt}}
\index[cmd]{\texttt{12pt}}

Diese drei Dokumentklassenoptionen definieren die Schriftgröße\index{Schriftgröße} fest.
Ist keine der drei Alternativen angeben, wird \LaTeX\ standardmäßig die Schriftgröße \verb!10pt! wählen.

\subsubsection{\texttt{onecolumn} oder \texttt{twocolumn}}
\index[cmd]{\texttt{onecolumn}}
\index[cmd]{\texttt{twocolumn}}
\index{Seitenlayout!einspaltig}\index{Einspaltig}
\index{Seitenlayout!zweispaltig}\index{Zweispaltig}

Diese beiden Optionen ermöglichen die Definition der Seitenformatierung. 
Zur Auswahl stehen einspaltiges und zweispaltiges Seitenlayout. 
Standardmäßig verwendet \LaTeX\ ein einspaltiges Seitenlayout.

\subsubsection{\texttt{oneside} oder \texttt{twoside}}
\index[cmd]{\texttt{oneside}}
\index[cmd]{\texttt{twoside}}

Hiermit kann die Seitenformatierung auf ein- oder doppelseitige Ausgabe festgelegt werden. 

Bei doppelseitiger Ausgabe sind die linken Seitenränder von geraden und ungeraden Seiten unterschiedlich groß, so dass beim doppelseitigen Ausdruck die Textfelder 
übereinstimmen. Bei einseitiger Ausgabe sind die Seitenränder von geraden und ungeraden Seiten gleich groß.

Standardmäßig ist bei den Dokumentklassen \verb!article!, \verb!report! und \verb!letter! die einseitige Ausgabe voreingestellt. Bei der Dokumentklasse \verb!book! hingegen ist doppelseitige Ausgabe voreingestellt.

\subsubsection{\texttt{titlepage} oder \texttt{notitlepage}}
\index[cmd]{\texttt{titlepage}}
\index[cmd]{\texttt{notitlepage}}
\index{Titelseite} 

Ob eine eigene Titelseite generiert wird oder ob der Titel horizontal oben zentriert auf der ersten Dokumentseite gesetzt ist, hängt von der verwendeten Dokumentklasse ab.

Bei den Dokumentklassen \verb!book! und  \verb!report! wird eine separate Titelseite erzeugt. Bei der Dokumentklasse \verb!article! ist das nicht der Fall. Mit den Optionen \verb!titlepage! und \verb!notitlepage! können Autoren dieses Verhalten ändern. 


\subsubsection{\texttt{draft} oder \texttt{final}}
\index[cmd]{\texttt{draft}}
\index[cmd]{\texttt{final}}
\index{Zeilenumbruch} 

Bei der Option \verb!draft! werden Zeilen,
bei denen \LaTeX\ den Zeilenumbruch nicht
korrekt realisieren konnte, das heißt Zeilen, die ein wenig über den Rand des 
Textfeldes hinausragen, mit einem fetten
schwarzen Balken am Rand gekennzeichnet.
Dieser Randbalken entfällt bei der
Option \verb!final!, die die Standardeinstellung ist.


\subsubsection{\texttt{leqno}}
\index[cmd]{\texttt{leqno}}

Ist diese Dokumentklassenoption
angeben, werden die Nummern 
von mathematischen Formeln nicht 
rechtsbündig, wie sonst automatisch 
üblich, sondern linksbündig gesetzt.

\subsubsection{\texttt{fleqn}}
\index[cmd]{\texttt{fleqn}}

Mit dieser Option werden abgesetzte mathematische Formeln nicht zentriert, sondern linksbündig gesetzt.




\subsubsection{\texttt{a4paper}, \texttt{a5paper}, \texttt{b5paper}, \texttt{letterpaper}, \texttt{legalpaper} und \texttt{executivepaper}}
\index[cmd]{\texttt{a4paper}}\index{DIN-A4}
\index[cmd]{\texttt{a5paper}}\index{DIN-A5}
\index[cmd]{\texttt{b5paper}}
\index[cmd]{\texttt{letterpaper}}
\index[cmd]{\texttt{legalpaper}}
\index[cmd]{\texttt{executivepaper}}


Diese Dokumentklassenoptionen definieren die Papiergröße\index{Papiergröße}.
Standardmäßig verwendet \LaTeX\ die Option \verb!letterpaper! als 
Standard-Papierformat. Da dieses US-amerikanische Papierformat in Mitteleuropa sehr unüblich ist, sollten Autoren im europäischen Raum mit der Option \verb!a4paper! auf DIN-A4 umzustellen. Eine Übersicht über die Dimensionen der unterschiedlichen Papierformate enthält Tabelle~\ref{Tabelle_Papierformate}.

\begin{table}[h!tb]
\centering
\caption{Papierformate}
\label{Tabelle_Papierformate}       % Give a unique label
\begin{tabular}{lrclrcl}
\hline
Dokumentklassenoption & \multicolumn{3}{c}{Größe [mm]} & \multicolumn{3}{c}{Größe [Zoll]} \\
\hline
\texttt{a4paper}        & 297   & x & 210   & 11,7 & x & 8,3  \\ 
\texttt{a5paper}        & 210   & x & 148   & 8,3  & x & 5,8  \\
\texttt{b5paper}        & 250   & x & 176   & 9,8  & x & 6,9  \\ 
\texttt{letterpaper}    & 279,4 & x & 215,9 & 11   & x & 8,5  \\
\texttt{legalpaper}     & 355,6 & x & 215,9 & 14   & x & 8,5  \\
\texttt{executivepaper} & 266,7 & x & 184,1 & 10,5 & x & 7,25 \\
\hline
\end{tabular}
\end{table}


\subsubsection{\texttt{landscape}}
\index[cmd]{\texttt{landscape}}

Mit der Option \verb!landscape! wird das Dokument im Querformat gesetzt.

\subsubsection{\texttt{openright} oder \texttt{openany}}
\index[cmd]{\texttt{openright}}
\index[cmd]{\texttt{openany}}

Bei der Dokumentklasse \verb!book!
beginnt ein neues Kapitel immer 
auf einer rechten (ungeraden) Seite. Wenn ein
vorhergegangenes Kapitel auf einer 
ungeraden Seite endet, dann wird eine leere,
gerade Seite eingefügt. Mit der 
Option \verb!openany! wird dieses Verhalten
von \LaTeX\ unterdrückt. Dann kann 
ein neues Kapitel auch auf einer geraden
Seite beginnen.

Bei der Dokumentklasse \verb!report! kann ein Kapitel
standardmäßig auf jeder Seite beginnen. 
Mit der Option \verb!openright! können Autoren
festlegen, dass Kapitel (genau wie bei 
der Klasse \verb!book!) nur auf ungeraden,
rechten Seiten beginnen dürfen.

\section{Seitenstile}
\index{Seitenstil}

Der grundsätzliche Aufbau einer Seite heißt
Seitenstil. Das Festlegen des Seitenstils geschieht mit dem Befehl
\verb!\pagestyle!\index[cmd]{\texttt{\textbackslash pagestyle}} in der Präambel der \verb!.tex!-Quelldatei. 
Der gewählte Seitenstil gilt dann für das gesamte Dokument.


\fbox{\texttt{\textbackslash pagestyle\{}\textsl{Seitenstil}\texttt{\}}}

\LaTeX\ bietet
einige vorgefertigte Seitenstile (siehe Tabelle~\ref{Tabelle_Seitenstile}).
Der
Seitenstil \texttt{plain} ist die Standardauswahl von
\LaTeX, wenn kein anderer
Seitenstil mit \texttt{\textbackslash pagestyle} ausgewählt ist.

\begin{table}[h!tb]
\centering
\caption{Seitenstile}
\label{Tabelle_Seitenstile}       % Give a unique label
\begin{tabular}{lp{10.2cm}}
\hline
Seitenstil & Aussehen \\
\hline
\texttt{empty} & Die Kopf- und Fußzeilen 
bleiben leer und es wird keine
Seitennummer gesetzt. Nur das Textfeld ist zu sehen.\\
\texttt{plain} & Die Kopfzeile bleibt 
leer. In der Fußzeile wird die
Seitennummer zentriert ausgegeben.  \\
\texttt{headings} & Die Kopfzeile
jeder Seite enthält 
die aktuelle Seitenzahl und eine Überschrift. Welche Überschrift
das genau ist, hängt von der verwendeten
Dokumentklasse ab.
Für gewöhnlich handelt es sich dabei
um die aktuelle Kapitel- oder Abschnittsüberschrift. 
Die Fußzeile bleibt leer
Die einzige Ausnahme ist die jeweils 
erste Seite eines neuen Kapitels. \\
\texttt{myheadings} & Der einzige
Unterschied zum Seitenstil \texttt{headings} ist, dass die Kopfzeile nicht automatisch 
gefüllt wird, sondern das der Autor den Inhalt der Kopfzeile mit den
Befehlen \texttt{\textbackslash markright} oder
\texttt{\textbackslash markboth} festlegen muss. \\
\hline
\end{tabular}
\end{table}

Es ist auch möglich, den Seitenstil einer einzelnen Seite festzulegen. Dieses geschieht nicht in der Präambel der \verb!.tex!-Quelldatei, sondern im Text des Dokuments. Der Befehl
\verb!\thispagestyle! legt den Seitenstil der
aktuellen Seite fest -- an der Stelle im 
Quelltext, an der der Befehl auftritt.


\fbox{\texttt{\textbackslash thispagestyle\{}\textsl{Seitenstil}\texttt{\}}}


\section{Definition der Kopfzeile mit \texttt{markright} und \texttt{markboth}}
\label{markright}
\index{Kopfzeile}
\index[cmd]{\texttt{\textbackslash markright}}
\index[cmd]{\texttt{\textbackslash markboth}}

Ist ein Dokument mit den Seitenstilen \verb!headings! oder \verb!myheadings! gesetzt, können Autoren mit Hilfe der 
Befehle \verb!\markright! und \verb!markboth! die Inhalte der Kopfzeilen im Dokument definieren und damit die Standardeinstellung (siehe Tabelle~\ref{Tabelle_Standardeinstellung_Kopfzeile}) von \LaTeX\ zu überschreiben. Die
Wirkung der beiden 
Befehle \texttt{\textbackslash markright} und \texttt{\textbackslash markboth} setzt erst aber der    
zweiten Seite des Dokuments ein.

Der Befehl \verb!\markright! eignet sich besonders dann, wenn das Dokument einseitig gesetzt wird (Dokumentklassenoption \verb!oneside!). Jede bedruckte Seite gilt dann als rechte Seite.

\fbox{\texttt{\textbackslash markright\{}\textsl{rechter Kopf}\texttt{\}}}

Für doppelseitig gesetzte Dokumente (Dokumentklassenoption \verb!twoside!) eignet sich \verb!\markboth!. Hier können Autoren die
Kopfzeilen für linke und rechte Seiten unterschiedlich definieren.

\fbox{\texttt{\textbackslash markboth\{}\textsl{linker Kopf}\texttt{\}\{}\textsl{rechter Kopf}\texttt{\}}}

Seiten mit \textsl{gerader} Seitennummer sind in diesem Fall \textsl{linke} Seiten und Seiten mit 
\textsl{ungerader} Seitennummer sind \textsl{rechte} Seiten. 

Zudem wird beim Befehl 
\verb!\markboth! die Seitennummer 
jeder Seite \textsl{linksbündig} auf jede 
Seite mit \textsl{gerader} Seitennummer 
gesetzt und \textsl{rechtsbündig} 
auf jede Seite mit \textsl{ungerader} Seitennummer.

Was bei \verb!\markboth! und \verb!\markright!
voreingestellt ist, also welche
Informationen die Kopfzeilen enthalten,
hängt von der verwendeten Dokumentklasse ab (siehe Tabelle~\ref{Tabelle_Standardeinstellung_Kopfzeile}). 

Bei doppelseitigem Druckstil und den 
Bearbeitungsklassen \verb!book! oder
\verb!report! wird von \verb!\markboth! 
automatisch die Überschrift des
aktuellen Kapitels (\texttt{\textbackslash chapter}) 
in die Kopfzeilen der linken 
Seiten gesetzt, und die
Kopfzeilen der
rechten Seiten enthalten 
die Überschriften der Abschnitte (\texttt{\textbackslash section}).

\begin{Verbatim}[frame=single]
\markboth{\chapter}{\section}
\end{Verbatim}

Bei einseitigem Druckstil und den Bearbeitungsklassen \verb!book! oder
\verb!article! wird von \verb!\markright! standardmäßig die 
Überschrift des aktuellen Kapitels (\texttt{\textbackslash chapter}) in die Kopfzeile der rechten 
Seiten gesetzt.

\begin{Verbatim}[frame=single]
\markright{\chapter}
\end{Verbatim}

Bei doppelseitigem Druckstil und
der Bearbeitungsklasse \verb!article!
wird von \verb!\markboth! die Überschrift des aktuellen Abschnitts (\texttt{\textbackslash section}) in die Kopfzeilen der linken 
Seiten gesetzt und die Überschrift
des aktuellen Unterabschnitts (\texttt{\textbackslash subsection})
wird in die Kopfzeilen der rechten Seiten geschrieben.

\begin{Verbatim}[frame=single]
\markboth{\section}{\subsection}
\end{Verbatim}

Bei einseitigem Druckstil und \verb!article!
wird von \verb!\markright! standardmäßig die 
Überschrift des aktuellen Abschnitts 
(\texttt{\textbackslash section}) in die Kopfzeile der rechten Seiten gesetzt.

\begin{Verbatim}[frame=single]
\markright{\section}
\end{Verbatim}


\begin{table}[h!tb]
\centering
\caption[Standardmäßiger Inhalt der Kopfzeile]{Standardmäßiger Inhalt der Kopfzeile~\cite{Kopka2000}}
\label{Tabelle_Standardeinstellung_Kopfzeile}       % Give a unique label
\begin{tabular}{llcc}
\hline
     &        & \multicolumn{2}{c}{Dokumentklasse} \\
Stil & Befehl & \texttt{book}, \texttt{report} & \texttt{article} \\
\hline
doppelseitig 
& \texttt{\textbackslash markboth\{}\textsl{linker Kopf}\texttt{\}}   
& \texttt{\textbackslash chapter} 
& \texttt{\textbackslash section} \\  
& \texttt{\textbackslash markboth\{}\textsl{rechter Kopf}\texttt{\}}   
& \texttt{\textbackslash section}
& \texttt{\textbackslash subsection} \\
einseitig 
& \texttt{\textbackslash markright}   
& \texttt{\textbackslash chapter}
& \texttt{\textbackslash section} \\
\hline
\end{tabular}
\end{table}

% Eine ähnliche Tabelle befindet sich bei Kopka2000 auf S.32

Alternativ zur automatischen Befüllung der Kopfzeilen mit den Überschriften der Kapitel, Abschnitten und Unterabschnitten, ist es auch möglich, einen beliebigen Text zu definieren, der auf allen Seiten in der Kopfzeile gesetzt wird. 

Wenn beispielsweise 
bei einem zweiseitigen Dokument mit
der Dokumentklasse \verb!book! auf jeder
linken Seite das Wort \textsl{Masterthesis} 
und auf jeder rechten Seite die
aktuelle Kapitelüberschrift
stehen soll, kann dieses mit den folgenden Befehlen realisiert werden:

\begin{Verbatim}[frame=single]
\pagestyle{headings}
\markright{Masterthesis}{\chapter}
\end{Verbatim}

Befinden sich auf einer Seite mehrere \texttt{\textbackslash section}-
oder \texttt{\textbackslash subsection}-Befehle, fügt \LaTeX\ standardmäßig auf \textsl{linken} Seiten 
die jeweils \textsl{letzte} und auf
\textsl{rechten} Seiten die jeweils 
\textsl{erste} Überschrift in die 
Kopfzeile ein.




\section{Definition der Kopfzeile mit \texttt{fancyhdr}}
\label{fancyheadings}
\index{fancyheadings}
\index{Kopfzeile}
\index{Fußzeile}

Umfangreichere Möglichkeiten zur Definition der Kopfzeile und der Fußzeile bietet das Erweiterungspaket \verb!fancyhdr!. Dieses definiert Befehle für den Seitenstil \verb!fancy!. Bei \verb!fancyhdr! sind, wie in den Abbildungen~\ref{Abbildungen_fancyhdr_gerade_Seiten} und \ref{Abbildungen_fancyhdr_ungerade_Seiten} zu sehen ist, die Kopf- bzw. Fußzeile dreigeteilt (in Links-Mitte-Rechts). 


\begin{figure}[H]
\centering
\fbox{
\begin{tabularx}{.75\textwidth}{XYZ}
\textsl{LK-gerade} & \textsl{MK-gerade} & \textsl{RK-gerade}\\
\hline
 &  & \\ 
 &  & \\ 
\multicolumn{3}{c}{{\LARGE Gerade Seiten}}\\
 &  & \\ 
 &  & \\ 
\hline
\textsl{LF-gerade} & \textsl{MF-gerade} & \textsl{RF-gerade}
\end{tabularx}
}
\caption[Aufteilung von \texttt{fancyheadings} bei geraden Seiten]{Aufteilung von \texttt{fancyheadings} bei geraden Seiten~\cite{GoossensMittelachSamarin2000}}
\label{Abbildungen_fancyhdr_gerade_Seiten}
\end{figure}





\begin{figure}[H]
\centering
\fbox{
\begin{tabularx}{.75\textwidth}{XYZ}
\textsl{LK-ungerade} & \textsl{MK-ungerade} & \textsl{RK-ungerade}\\
\hline
 &  & \\ 
 &  & \\ 
\multicolumn{3}{c}{{\LARGE Ungerade Seiten}}\\
 &  & \\ 
 &  & \\ 
\hline
\textsl{LF-ungerade} & \textsl{MF-ungerade} & \textsl{RF-ungerade}
\end{tabularx}
}
\caption[Aufteilung von \texttt{fancyheadings} bei ungeraden Seiten]{Aufteilung von \texttt{fancyheadings} bei ungeraden Seiten~\cite{GoossensMittelachSamarin2000}}
\label{Abbildungen_fancyhdr_ungerade_Seiten}
\end{figure}


Um \verb!fancyhdr! zu verwenden, muss das Erweiterungspaket
in der Präambel der \LaTeX-Quelldatei mit dem Befehl \verb!\usepackage{fancyheadings}! eingebunden und der Seitenstil \verb!fancy! mit dem Befehl \verb!\pagestyle{fancy}! ausgewählt sein.

\begin{Verbatim}[frame=single]
\pagestyle{fancy}
\end{Verbatim}

Die Definition des Inhalts der in den Abbildungen~\ref{Abbildungen_fancyhdr_gerade_Seiten} und \ref{Abbildungen_fancyhdr_ungerade_Seiten} dargestellten Felder der Kopf- bzw. Fußzeilen auf den gerade und ungeraden Seiten geschieht mit den Befehlen \texttt{\textbackslash lhead}, \texttt{\textbackslash chead}, \texttt{\textbackslash rhead}, \texttt{\textbackslash lfoot}, \texttt{\textbackslash cfoot} und \texttt{\textbackslash rfoot}.


\begin{boxedminipage}{\textwidth}
\texttt{\textbackslash lhead[}\textsl{LK-gerade}\texttt{]\{}\textsl{LK-ungerade}\texttt{\}} \\
\texttt{\textbackslash chead[}\textsl{MK-gerade}\texttt{]\{}\textsl{MK-ungerade}\texttt{\}} \\
\texttt{\textbackslash rhead[}\textsl{RK-gerade}\texttt{]\{}\textsl{RK-ungerade}\texttt{\}} \\
\texttt{\textbackslash lfoot[}\textsl{LF-gerade}\texttt{]\{}\textsl{LF-ungerade}\texttt{\}} \\
\texttt{\textbackslash cfoot[}\textsl{MF-gerade}\texttt{]\{}\textsl{MF-ungerade}\texttt{\}} \\
\texttt{\textbackslash rfoot[}\textsl{RF-gerade}\texttt{]\{}\textsl{RF-ungerade}\texttt{\}} 
\end{boxedminipage}


Diese Befehle müssen in der Präambel der \LaTeX-Quelldatei, also vor dem 
\verb!\begin{document}! stehen. Wie in den Abbildungen~\ref{Abbildungen_fancyhdr_gerade_Seiten} und \ref{Abbildungen_fancyhdr_ungerade_Seiten} dargestellt, werden die Informationen in den Feldern \textsl{LK} und \textsl{LF} linksbündig und in den Feldern \textsl{RK} und \textsl{RF} rechtsbündig gesetzt. Die Einträge in den Feldern \textsl{MK} und \textsl{MF} erscheinen zentriert. 


Das Aussehen der Linen unter der Kopfzeile und über der Fußzeile wird mit den 
Befehlen \verb!\headrulewidth!\index[cmd]{\texttt{\textbackslash headrulewidth}}
und \verb!\footrulewidth!\index[cmd]{\texttt{\textbackslash footrulewidth}} 
beeinflusst. Auch diese beiden Befehle müssen in der Präambel der \LaTeX-Quelldatei stehen.


\begin{boxedminipage}{\textwidth}
\texttt{\textbackslash renewcommand\{\textbackslash headrulewidth\}}\{\textsl{Wert}\} \\
\texttt{\textbackslash renewcommand\{\textbackslash footrulewidth\}}\{\textsl{Wert}\} 
\end{boxedminipage}
\index[cmd]{\texttt{\textbackslash renewcommand}}

Sollen die Linen unter der Kopfzeile und über der Fußzeile gar nicht erscheinen, so kann die Strichbreite einfach auf einen Wert \verb!0pt! gesetzt werden.

Die Höhe der Kopfzeile kann mit dem Längenbefehl \verb!\headheight!\index[cmd]{\texttt{\textbackslash headheight}} 
beeinflusst werden. So ist es möglich mehrzeilige Kopfzeilen zu setzen.

\fbox{\texttt{\textbackslash setlength\{\textbackslash headheight\}}\{\textsl{Wert}\}}



Abbildung~\ref{Abbildungen_fancyhdr_Standardeinstellung} zeigt das voreingestellte Seitenlayout des Erweiterungspaketes \verb!fancyhdr!.

\begin{figure}[H]
\begin{minipage}[t]{0.48\textwidth}
\centering
\fbox{
\begin{tabular}{lcr}
\texttt{\textbackslash rightmark} & & \texttt{\textbackslash leftmark}\\
\hline
& & \\ 
& & \\ 
\multicolumn{3}{c}{{\LARGE Gerade Seiten}}\\
& & \\ 
& & \\ 
\multicolumn{3}{c}{\texttt{\textbackslash thepage}}
\end{tabular}
}
\end{minipage}
\hfill
\begin{minipage}[t]{0.48\textwidth}
\centering
\fbox{
\begin{tabular}{lcr}
\texttt{\textbackslash leftmark} & & \texttt{\textbackslash rightmark}\\
\hline
& & \\ 
& & \\ 
\multicolumn{3}{c}{{\LARGE Ungerade Seiten}}\\
& & \\ 
& & \\ 
\multicolumn{3}{c}{\texttt{\textbackslash thepage}}
\end{tabular}
}
\end{minipage}
\caption[Voreingestelltes Seitenlayout bei \texttt{fancyhdr}]{Voreingestelltes Seitenlayout bei \texttt{fancyhdr}~\cite{GoossensMittelachSamarin2000}}
\label{Abbildungen_fancyhdr_Standardeinstellung}
\end{figure}

Der Befehl \verb!\thepage! fügt an der Stelle, an der er eingefügt ist, auf jeder Seite die entsprechende Seitenzahl ein.

Der Befehl \verb!\leftmark! fügt bei den Dokumentklassen \verb!book! und \verb!report! standardmäßig die Überschrift des aktuellen Kapitels (\verb!chapter!) auf der jeweiligen Seite ein. \verb!\rightmark! fügt standardmäßig die Überschrift des aktuellen Abschnitts (\verb!section!) auf der jeweiligen Seite ein.

Bei Dokumentklassen, die über keine Kapitel verfügen, wie zum Beispiel \verb!article!, fügt  \verb!\leftmark! die Überschrift des aktuellen Abschnitts und \verb!\rightmark! die Überschrift des aktuellen Unterabschnitts ein.

\section{Seitennummerierung}
\index{Seitennummerierung}

\LaTeX\ nummeriert automatisch 
die Seiten eines Dokuments durch. Der voreingestellte
Stil der Seitennummerierung ist dabei 
\verb!arabic!, also die Nummerierung mit
arabischen Ziffern.\index{Ziffern!arabisch}\index[cmd]{\texttt{arabic}}

Der Befehl \verb!\pagenumbering!\index[cmd]{\texttt{\textbackslash pagenumbering}} ermöglicht die Definition des Seitennummerierungsstils.\index{Seitennummerierung!Stil}

\fbox{\texttt{\textbackslash pagenumbering\{\textsl{Stil}\}}}

Weitere, von \LaTeX\ unterstützte
Stile zur Seitennummerierung (siehe Tabelle~\ref{Tabelle_Nummerierungsstilarten})
sind die Nummerierung mit kleinen (\verb!roman!) und großen (\verb!Roman!) römischen Ziffern\index{Ziffern!römisch} 
oder mit fortlaufenden Klein- oder 
Großbuchstaben von a-z (\verb!alph!) bzw. A-Z (\verb!Alph!).



\begin{table}[h!tb]
\centering
\caption{Seitennummerierungsstile}
\label{Tabelle_Nummerierungsstilarten}       % Give a unique label
\begin{tabular}{ll}
\hline
Stil & Ergebnis \\
\hline
\texttt{arabic} & Nummerierung mit arabischen Ziffern \\
\texttt{roman} & Nummerierung in kleinen römischen Ziffern \\
\texttt{Roman} & Nummerierung in großen römischen Ziffern \\
\texttt{alph} & Nummerierung in Kleinbuchstaben (a-z) \\
\texttt{Alph} & Nummerierung in Großbuchstaben (A-Z) \\
\hline
\end{tabular}
\end{table}

Bei den Seitennummerierungsstilen \verb!alph! und \verb!Alph!
entsprechen die Buchstaben a-z bzw. A-Z den Zählerwerten 1-26 des Zählers \verb!page! für
die Seitenzahlen. Das bedeutet, auch wenn die Seitenzahlen in Buchstaben
angezeigt werden, rechnet \LaTeX\ intern ganz normal mit Zahlen weiter. 

Eine Änderung des Seitennummerierungsstils 
mit \verb!\pagenumbering! setzt den \LaTeX-internen Zähler der 
Seitennummern \verb!page!\index[cmd]{\texttt{\textbackslash page}} automatisch 
auf den Wert \verb!1! zurück.

Um den Startwert der Seitennummerierung 
auf einen anderen Wert als \texttt{1} zu setzen, muss der Zähler
\texttt{page} auf einen höheren 
Wert gesetzt werden. 

\fbox{\texttt{\textbackslash setcounter\{page\}\{\textsl{Nummer}\}}}
\index[cmd]{\texttt{\textbackslash setcounter}}


Sollen beispielsweise das Vorwort und das Inhaltsverzeichnis eines Dokuments mit (großen) römischen Ziffern 
und das eigentliche 
Dokument mit arabischen Ziffern durchnummeriert werden, kann dieses auf folgende Art und Weise realisiert werden: 
Nach dem \verb!\begin{document}! wird mit \verb!\pagenumbering{Roman}! der 
Seitennummerierungsstil auf große römische Ziffern festgelegt und direkt nach 
dem ersten \texttt{\textbackslash chapter}-Befehl erfolgt mit \verb!\pagenumbering{arabic}! ein Wechsel 
auf arabische Ziffern. Die Struktur des Dokuments ist also so oder so ähnlich:


\begin{Verbatim}[frame=single]
<Präambel der LaTeX-Quelldatei>
\begin{document}
\pagenumbering{Roman}
...
\tableofcontents
...
\chapter{Kapitel 1}
\pagenumbering{arabic}
...
\end{Verbatim}




\section{Wichtige Abstände}
\index{Abstände}

In diesem Abschnitt ist die Anpassung wichtiger Abstände im Dokument beschrieben. Bei diesen handelt es sich um den Zeilenabstand, den Absatzabstand und die Tiefe der Einrückung der ersten Zeile eines Absatzes.


\subsection{Zeilenabstand}
\index{Zeilenabstand} 

Der Abstand zwischen zwei aufeinander 
folgenden Zeilen heißt 
Zeilenabstand. Diesen definiert der Wert in der 
Variablen \verb!\baselineskip!\index[cmd]{\texttt{\textbackslash baselineskip}}.
Der \LaTeX-Compiler legt den Zeilenabstand 
automatisch fest und berücksichtigt dabei die
aktuelle Schriftgröße. 

Um den Zeilenabstand zu ändern, sollte nicht \verb!\baselineskip! einen neuen Wert zugewiesen bekommen. Gründe dafür sind:


\begin{itemize}
\item Der Wert für \verb!\baselineskip! 
kann nicht in der Präambel definiert werden.
\item Bei jeder Änderung der Schriftgröße wird 
eine selbst vorgenommene Änderung des Zeilenabstandes
überschrieben, da eine Änderung der 
Schriftgröße auch eine Änderung des
Zeilenabstandes von \LaTeX\ nach sich zieht.
\item Wird \verb!\baselineskip! innerhalb 
eines Absatzes geändert, gilt
das rückwirkend für den gesamten Absatz.
\item Wird innerhalb eines Absatzes 
mehrmals mit \verb!\baselineskip!
der Zeilenabstand geändert, gilt nur die 
letzte Änderung im Absatz, da diese alle anderen Änderungen rückwirkend überschreibt.
\end{itemize}

Besser als eine Änderung des Werts von \verb!\baselineskip! ist es, den Längenbefehl 
\verb!\baselinestretch! zu verwenden. Dieser enthält
einen Faktor, mit 
dem der normale
Zeilenabstand (also \verb!\baselineskip!)
multipliziert wird. 
Standardmäßig hat \verb!\baselinestretch!\index[cmd]{\texttt{\textbackslash baselinestretch}}
den Wert \verb!1!. Die Zuweisung eines neuen Werts geschieht folgenderweise:

\fbox{\texttt{\textbackslash renewcommand\{\textbackslash baselinestretch\}\{\textsl{Faktor}\}}}
\index[cmd]{\texttt{\textbackslash renewcommand}}

Wird der Wert des Längenbefehls \verb!\baselinestretch! 
in der Präambel geändert, gilt
diese Änderung für das gesamte Dokument.

Wird der Wert von \verb!\baselinestretch! außerhalb der Präambel geändert, dann gilt die Änderung des Zeilenabstandes erst nach dem nächsten 
Schriftgrößenwechsel. Darum macht es 
Sinn, direkt nach dem Festlegen des neuen 
Zeilenabstandes einen Schriftgrößenbefehl 
zu bringen. Im einfachsten Fall, wenn die Schriftgröße gleich bleiben soll, geschieht das mit dem Befehl \verb!\normalsize!.



\subsection{Absatzabstand}
\index{Absatzabstand}
\label{Absatzabstand}

Der Abstand, den der \LaTeX-Compiler zwischen zwei Absätzen einfügt, hängt vom Wert des 
Längenbefehls \verb!\parskip!\index[cmd]{\texttt{\textbackslash parskip}} ab. 

Es ist ratsam, für die Definition des Absatzabstands die Maßeinheit \verb!ex! (siehe Seite~\pageref{Tabelle_Masseinheiten}) zu verwenden, da diese sich an der aktuellen Zeichengröße orientiert und den
Absatzabstand als elastisches Maß (siehe Seite~\pageref{Elastische_Masse}) zu definieren, damit \LaTeX\ das entstehen von 
\textsl{Hurenkindern}\index{Hurenkind} und \textsl{Schusterjungen}\index{Schusterjunge} verhindert. Mit diesen Begriffen bezeichnet man den Makel beim Textsatz, dass eine Seite mit der ersten Zeile eines 
Absatzes endet (so genannte \textsl{Schusterjungen}) oder aber eine neue 
Seite mit der letzten Zeile eines Absatzes beginnt (so genannte
\textsl{Hurenkinder}).

Die folgende Zeile beispielsweise setzt den Absatzabstand
auf \verb!1.5ex! plus/minus \verb!0.5ex!:

\begin{Verbatim}[frame=single]
\setlength{\parskip}{1.5ex plus 0.5ex minus 0.5ex}
\end{Verbatim}




\subsection{Einrückung der ersten Zeile eines Absatzes}
\index{Einrückung!Absatz}


\setlength{\parindent}{1.5em}
Bei englischsprachigen Dokumenten ist es üblich, keinen zusätzlichen Freiraum zwischen Absätzen einzufügen, und dafür (außer beim ersten Absatz) immer die erste Zeile eines Absatzes um einen festen Abstand (z.B. einen halben Zentimeter) einzurücken.

Auch \LaTeX\ nimmt diese Einrückung
standardmäßig vor. Wie es aussieht, wenn immer die erste 
Zeile eines Absatzes eingerückt wird, zeigt dieser Abschnitt. 
Dabei ist die erste Zeile immer um \verb!1.5em! eingerückt.
Der Betrag, um den \LaTeX\ die erste 
Zeile eines jeden Absatzes nach rechts 
einrückt, ist im Längenbefehl \verb!\parindent!
\index[cmd]{\texttt{\textbackslash parindent}} gespeichert.

Das Einrückverhalten von \LaTeX\ können Autoren anpassen, indem sie mit
dem Befehl \verb!\setlength! dem
Längenbefehl \verb!\parindent!
einen anderen
Wert zuweisen. Die Syntax ist dabei folgende:


\fbox{\texttt{\textbackslash setlength\{\textbackslash parindent\}\{}\textsl{Betrag}\texttt{\}}}

Um das automatische
Einrücken zu unterbinden, genügt es also, dem Längenbefehl \verb!\parindent! in der Präambel der \LaTeX-Quelldatei den Wert \verb!0em! zuzuweisen.

Um nur das Einrückverhalten des nächsten Absatzes zu beeinflussen, existiert der Befehl \verb!\noindent!. Dieser verhindert das Einrücken des nächsten Absatzes, habt aber auch nur für den nächsten Absatz Gültigkeit.

\fbox{\texttt{\textbackslash noindent}}

Zudem existiert 
der Befehl \verb!\indent!\index[cmd]{\texttt{\textbackslash indent}}, der das 
Einrücken nur des nächsten Absatzes anweist.

\fbox{\texttt{\textbackslash indent}}

\setlength{\parindent}{0em}

\section{Parameter zur Seitendeklaration}
\index{Seitendeklaration}

Dieser Abschnitt beschreibt einige Längenbefehle (siehe Tabelle~\ref{Tabelle_Längenbefehle_Layout}), die die geometrischen Dimensionen des Layouts beeinflussen.
Diese Befehle sind nur dann für Autoren relevant, wenn sie von den 
Standardwerten, die von der verwendeten
Dokumentklasse abhängen, abweichen wollen. 

Jede Seite besteht aus einer Kopfzeile\index{Kopfzeile} (\textsl{Head}), einem Textkörper\index{Textkörper}
(\textsl{Body}) und einer Fußzeile\index{Fußzeile} (\textsl{Foot}).

\begin{table}[h!tb]
\centering
\caption[Längenbefehle zur Beeinflussung des Layouts]{Längenbefehle zur Beeinflussung des Layouts~\cite{GoossensMittelachSamarin2000}}
\label{Tabelle_Längenbefehle_Layout}       % Give a unique label
\begin{tabularx}{\textwidth}{lX}
\hline
Befehl & Bedeutung \\
\hline
\texttt{\textbackslash columnsep} & Abstand der Spalten 
bei Mehrspaltensatz \\
\texttt{\textbackslash columnseprule} & Breite der Trennline, die die
Spalten beim Mehrspaltensatz voneinander trennt (normalerweise \texttt{0pt},
also unsichtbar) \\
\texttt{\textbackslash columnwidth} & Spaltenbreite beim Mehrspaltensatz.
Hängt von \texttt{\textbackslash textwidth} und 
\texttt{\textbackslash columnsep} ab\\
\texttt{\textbackslash evensidemargin} & Zusätzlicher linker Rand auf geraden
Seiten (bei zweiseitigem Satz)\\
\texttt{\textbackslash footskip} & Vertikale Abstand zwischen der 
Grundlinie der letzten Textzeile und der Grundlinie der Fußzeile\\
\texttt{\textbackslash headheight} & Höhe der Kopfzeile\\
\texttt{\textbackslash headsep} & Vertikale Abstand zwischen Kopfzeile und
Textkörper\\
\texttt{\textbackslash linewidth} & Breite der aktuellen Zeile\\
\texttt{\textbackslash marginparpush} & Vertikale Mindestabstand zwischen
zwei aufeinander folgenden Marginalien\\
\texttt{\textbackslash marginparsep} & Horizontaler Abstand zwischen 
Textkörper und Marginalien\\
\texttt{\textbackslash marginparwidth} & Breite der Marginalien \\
\texttt{\textbackslash oddsidemargin} & Zusätzlicher linker Rand auf ungeraden
Seiten (bei zweiseitigem Satz)\\
\texttt{\textbackslash paperheight} & Höhe des Papiers \\
\texttt{\textbackslash paperwidth} & Breite des Papiers \\
\texttt{\textbackslash topmargin} & Abstand vom oberen Rand des Papiers
bis zum oberen Rand der Kopfzeile\\
\texttt{\textbackslash textheight} & Höhe des Textkörpers
ohne Kopf- und Fußzeile\\
\texttt{\textbackslash textwidth} & Breite des Textkörpers \\
\hline
\end{tabularx}
\index[cmd]{\texttt{\textbackslash columnsep}}
\index[cmd]{\texttt{\textbackslash columnseprule}}
\index[cmd]{\texttt{\textbackslash columnwidth}}
\index[cmd]{\texttt{\textbackslash evensidemargin}}
\index[cmd]{\texttt{\textbackslash footskip}}
\index[cmd]{\texttt{\textbackslash headheight}}
\index[cmd]{\texttt{\textbackslash headsep}}
\index[cmd]{\texttt{\textbackslash linewidth}}
\index[cmd]{\texttt{\textbackslash marginparpush}}
\index[cmd]{\texttt{\textbackslash marginparsep}}
\index[cmd]{\texttt{\textbackslash marginparwidth}}
\index[cmd]{\texttt{\textbackslash oddsidemargin}}
\index[cmd]{\texttt{\textbackslash paperheight}}
\index[cmd]{\texttt{\textbackslash paperwidth}}
\index[cmd]{\texttt{\textbackslash topmargin}}
\index[cmd]{\texttt{\textbackslash textheight}}
\index[cmd]{\texttt{\textbackslash textwidth}}
\end{table}

Das Zuweisen neuer Werte zu den Längenbefehlen in Tabelle~\ref{Tabelle_Längenbefehle_Layout} erfolgt mit dem Befehl \verb!\setlength! (siehe Seite~\pageref{Absatzabstand}).




Das Erweiterungspaket \verb!layouts.sty! ermöglicht die Ausgabe einer übersichtlichen Darstellung des aktuellen Seitenlayouts. Um eine Darstellung, wie in Abbildung~\ref{Abbildung_Ausgabe_layout} zu auszugeben, lassen Sie mit dem Befehl \verb!\currentpage!\index[cmd]{\texttt{\textbackslash currentpage}} 
die aktuellen Werte der Längenbefehle auslesen, 
legen mit dem Befehl \verb!\oddpagelayouttrue!\index[cmd]{\texttt{\textbackslash oddpagelayouttrue}} oder alternativ mit dem Befehl
\verb!\oddpagelayoutfalse!\index[cmd]{\texttt{\textbackslash oddpagelayoutfalse}} fest, 
ob Sie das Layout einer ungeraden oder einer geraden Seiten ausgeben wollen und setzen die Übersicht mit dem Befehl \verb!\pagedesign!.


\begin{figure}[htbp]
\currentpage
\oddpagelayouttrue
\pagedesign
\caption{Layout ungerader Seiten im aktuellen Dokument} \label{Abbildung_Ausgabe_layout}
\end{figure}





\section{Mehrspaltenlayout}

Soll ein Dokument ab einer bestimmten Stelle zweiseitig gesetzt werden, muss dafür an der gewünschten Stelle der Befehl \verb!\twocolumn! aufgerufen werden.

\index{Seitenlayout!zweispaltig}\index{Zweispaltig}

An der Stelle, wo der Befehl aufgerufen wird, beginnt \LaTeX\ eine neue Seite mit zweispaltigem Layout. Ist eine Überschrift gewünscht, die über beiden Spalten steht, wird diese als optionales Argument in eckigen Klammern hinter dem Befehl \verb!\twocolumn!
angegeben. 

\fbox{\texttt{\textbackslash twocolumn[Überschrift über beide Spalten]}}
\index[cmd]{\texttt{\textbackslash twocolumn}}

Eine Eigenschaft von \verb!\twocolumn! ist, dass es immer dafür sorgt, dass ganze Seiten zweispaltig gesetzt werden. Dabei wird immer zuerst die linke 
Spalte und dann die rechte Spalte gefüllt. 
So kann es zu einer sehr unausgewogenen 
Verteilung des Textes kommen.

Soll wieder auf ein einspaltiges Layout umgeschaltete werden, geschieht dieses mit dem Befehl \verb!\onecolumn!.

\index{Seitenlayout!einspaltig}\index{Einspaltig}

\fbox{\texttt{\textbackslash onecolumn}}
\index[cmd]{\texttt{\textbackslash onecolumn}}




\section{Querformat}
\index{Querformat}
\index{Seitenlayout!Querformat}
\index{Hochformat} 
\index{Seitenlayout!Hochformat}
\index{Portrait} 
\index{Seitenlayout!Portrait}

Beim Setzen von Text ist es üblich,
das die längere Papierseite die vertikale
ist und die kürzere die horizontale.
Dieses Konzept heißt \textsl{Hochformat} 
oder \textsl{Portrait}. Für die 
allermeisten Arten von Dokumenten ist 
dieses das übliche Vorgehen.

Für einige Anwendungen,  zum Beispiel wenn Aushänge mit breiten Bildern oder Tabellen gesetzt werden sollen, empfiehlt es sich allerdings
die längere Papierseite als
die Horizontale zu verwenden. In diesem Fall spricht man vom \textsl{Querformat}.

Um Seiten im Querformat zu setzen, ist das Makropaket
\verb!portland! nützlich. Dieses definiert die Befehle 
\verb!\landscape!\index[cmd]{\texttt{\textbackslash landscape}} und
\verb!\portrait!\index[cmd]{\texttt{\textbackslash portrait}} sowie die beiden Umgebungen:

\begin{itemize}
\item \verb!\begin{portrait}!\dots\verb!\end{portrait}! für das Hochformat und\index[cmd]{\texttt{portrait}}
\item \verb!\begin{landscape}!\dots\verb!\end{landscape}! für das Querformat\index[cmd]{\texttt{landscape}}
\end{itemize}



Der Befehl \verb!\portrait! setzt das 
Seitenlayout auf die ursprünglichen Werte
des Hochformats, so wie sie zu
Beginn eines Dokuments nach dem
\verb!\begin{document}! vorgegeben sind.
Beim Befehl \verb!\landscape! werden 
die horizontalen und vertikalen Werte
vertauscht. Das Textfeld belegt also im Landscape-Modus die 
gleiche Fläche und die gleichen
Koordinaten wie im Portrait-Modus. 


An der Stelle im Quelltext, wo der  Befehl \verb!\portrait! oder alternativ der Befehl \verb!\landscape! aufgerufen wird, beginnt der \LaTeX-Compiler eine neue Seite, die dann entweder
im Hoch- oder Querformat erscheint. Die beiden Befehle \verb!\portrait! und \verb!\landscape! gelten aber immer nur für die nächste Seite. Sollen
mehrere aufeinander folgende Seiten im Querformat gesetzt werden,  
ist es bequemer, die Umgebung \verb!\landscape! zu 
nutzen, als für jede Seite den Befehl \verb!\landscape! aufzurufen.



\section{Befehle und Umgebungen zur Gliederung}
\index{Gliederung}


Die Gliederung von Dokumenten unterstützt deren Lesbarkeit. Dieser Abschnitt stellt Befehle und Umgebungen zur Realisierung der Titelseite, der Zusammenfassung (Abstract), von Kapiteln, Abschnitten und Unterabschnitten, sowie Anhängen vor.


\subsection{Titelei}
\index{Titelei}

Eine Titelseite kann unter \LaTeX\ entweder komplett manuell definiert, oder komfortabel mit Hilfe weniger Befehle generiert werden. Die zweite Alternative geht deutlich schneller, bietet dafür aber auch weniger Freiheiten als die erste Alternative.

Soll die Titelseite automatisch generiert werden, geschieht dieses mit Hilfe der Befehle 
\verb!\maketile!\index[cmd]{\texttt{\textbackslash maketile}}, \verb!\title!, \verb!\author!, \verb!\date! und \verb!\thanks!.

Der Befehl \verb!\title!\index[cmd]{\texttt{\textbackslash title}} dient zur Definition des Dokumenttitels. 

\fbox{\texttt{\textbackslash title\{}\textsl{Text}\texttt{\}}}

Mit dem Befehl \verb!\author!\index[cmd]{\texttt{\textbackslash author}} können Name und Adresse des Autors oder 
auch mehrerer Autoren definiert werden.

\fbox{\texttt{\textbackslash author\{}\textsl{Text}\texttt{\}}}

Gibt es mehrere Autoren, so ist es möglich,
deren Namen mit dem Befehl \verb!\and!\index[cmd]{\texttt{\textbackslash and}} zu verknüpfen. Sollen beispielsweise die Namen und Adressen von zwei Autoren auf dem Titel erscheinen, ist eine Lösung wie die folgende denkbar:


\begin{Verbatim}[frame=single]
\author{erster Autor \\Adresse des \\ersten Autors \and
zweiter Autor \\Adressse des \\zweiten Autors
\end{Verbatim}

Auf der Titelseite würde dieses Beispiel zu folgendem Ergebnis führen:

\begin{center}
% \fbox{
\begin{tabular}{p{1cm}cp{3cm}cp{1cm}}
& erster Autor  & & zweiter Autor   & \\
& Adresse des   & & Adresse des     & \\
& ersten Autors & & zweiten Autors  & \\
\end{tabular}
% }
\end{center}

Der Befehl \verb!\date! setzt einen
zentrierten Textblock unterhalb der Angaben zu
den Autoren. Dieser Textblock kann ein Datum oder sonst eine Information enthalten.

\fbox{\texttt{\textbackslash date\{}\textsl{Datum}\texttt{\}}}

Eine einfache Möglichkeit, das aktuelle Datum einzufügen, ist die Verwendung der Befehlskombination \verb!\date{\today}!\index[cmd]{\texttt{\textbackslash date}}, 
denn der Befehl \verb!\today!\index[cmd]{\texttt{\textbackslash today}}
fügt an der Stelle, an der er aufgerufen wird, das aktuelle Datum ein.

Fehlt der Befehl \verb!\date!, fügt \LaTeX\ automatisch das aktuelle Datum auf die Titelseite ein. Ist keine Datumsangabe auf der Titelseite gewünscht, sollte der Befehl \verb!\date! ohne Argumente, also mit leeren geschweiften Klammern im Quelltext stehen.
Das unterdrückt die automatische Datumsangabe durch \LaTeX.

Der Befehl \verb!\thanks!
\index[cmd]{\texttt{\textbackslash thanks}} setzt auf der Titelseite eine einzeilige Fußnote.

\fbox{\texttt{\textbackslash thanks\{}\textsl{Text}\texttt{\}}}

Eine häufige Anwendungen von \verb!\thanks! ist, wenn mehrere Autoren an einem Dokument gearbeitet haben und weitere Informationen zu deren Beitrag zum Dokument, zu deren Erreichbarkeit oder eventuell zu deren Arbeitgebern auf der Titelseite erwähnt sein sollen. 

Der Befehl \verb!\maketitle!\index[cmd]{\texttt{\textbackslash maketile}}, 
der die Titelseite nach den gemachten Vorgaben setzt, sollte sich im Quelltext nach den Befehlen \verb!\title!, \verb!\author!, \verb!\date!, \verb!\thanks! und nach dem \verb!\begin{document}! befinden.

\fbox{\texttt{\textbackslash maketile}}


\subsection{Die Zusammenfassung -- das Abstract}
\index{Abstract}
\index{Zusammenfassung}

Als \textsl{Abstract} bezeichnet man einen (kurzen) Text, der den Inhalt des
Dokuments zusammenfasst. 
Es steht abhängig von der Dokumentklasse 
auf der Titelseite, beidseitig eingerückt unter dem Titel oder auf einer
eigenen Seite nach der Titelseite. 

Um solche Zusammenfassungen unter 
\LaTeX\ zu setzen, existiert die
Umgebung \verb!abstract!\index[cmd]{\texttt{abstract}}.

\fbox{\texttt{\textbackslash begin\{abstract\}\texttt{\dots}\textbackslash end\{abstract\}}}

In der Dokumentklasse \verb!book! ist
die Umgebung \verb!abstract!
nicht definiert.
Eine Zusammenfassung am Anfang eines 
Buches wäre auch sehr ungewöhnlich. Wenn,
dann finden sich solche Zusammenfassungen 
auf dem Cover eines Buches oder in
Buchbesprechungen, und diese werden 
nicht in der Klasse \verb!book! erstellt~\cite{Kopka2000}.

% So wird das auch bei Kopka2000 auf Seite 39 argumentiert

Die Dokumentenklasse \verb!article! hat standardmäßig keine extra Seite für die Titelseite\index{Titelseite}.
Aus diesem Grund setzt der \LaTeX-Compiler bei \verb|article|
die Zusammenfassung auf der ersten
Seite unter der Überschrift. Wurde allerdings die 
Option \verb!titlepage! bei der Deklaration der Dokumentklasse verwendet,
gibt es doch eine eigene Seite für die Titelseite und die Zusammenfassung 
erscheint auf der Titelseite unter dem Titel. 
beidseitig eingerückt und in einer kleinen Schriftgröße.


Bei der Dokumentklasse \verb!report!
erscheint die Zusammenfassung 
auf einer separaten Seite ohne
Seitennummer nach der Titelseite. 
Die Schriftgröße der Zusammenfassung entspricht der Standardgröße. Eine
beidseitige Einrückung wie bei der 
Dokumentklasse \verb!article! erfolgt nicht.




\subsection{Gliederungsbefehle}
\index{Gliederungsbefehl}

\LaTeX\ enthält für die 
fortlaufende Untergliederung von Dokumenten in Teile, 
Kapitel, Abschnitte, Paragraphen usw. entsprechende
Befehle.
Tabelle~\ref{Tabelle_Gliederungsebenen} enthält eine 
Zusammenstellung der unter \LaTeX\
verfügbaren Gliederungsbefehle 
und der zugehörigen Werte des Zählers
\verb!tocdepth!\index[cmd]{\texttt{\textbackslash tocdepth}}.


\begin{table}[h!tb]
	\centering
	\caption[Gliederungsebenen und der Zähler \texttt{tocdepth}]{Gliederungsebenen und der Zähler \texttt{tocdepth}~\cite{Schlosser2009}}
	\label{Tabelle_Gliederungsebenen}       % Give a unique label
	\begin{tabular}{ll}
		\hline
		Gliederungsebene                      & Wert \\
		\hline
		\texttt{\textbackslash part}          & -1 bei \texttt{book} und \texttt{report}, 0 bei \texttt{article}  \\
		\texttt{\textbackslash chapter}       & 0    \\
		\texttt{\textbackslash section}       & 1    \\
		\texttt{\textbackslash subsection}    & 2    \\
		\texttt{\textbackslash subsubsection} & 3    \\
		\texttt{\textbackslash paragraph}     & 4    \\
		\texttt{\textbackslash subparagraph}  & 5    \\
		\hline
	\end{tabular}
	\index[cmd]{\texttt{\textbackslash part}}
	\index[cmd]{\texttt{\textbackslash chapter}}
	\index[cmd]{\texttt{\textbackslash section}}
	\index[cmd]{\texttt{\textbackslash subsection}}
	\index[cmd]{\texttt{\textbackslash subsubsection}}
	\index[cmd]{\texttt{\textbackslash paragraph}}
	\index[cmd]{\texttt{\textbackslash subparagraph}}
\end{table}



Alle diese Befehle zur Untergliederung
von Dokumenten, mit
Ausnahme von
\texttt{\textbackslash part} (\textsl{Teil}\index{Teil}), bauen eine
Gliederungshierarchie auf.
Bei den Dokumentklassen 
\verb!book! und
\verb!report! beginnt die 
Gliederung mit
\textsl{Kapiteln}\index{Kapitel} (\texttt{\textbackslash chapter}). Die
Kapitel werden untergliedert in
\textsl{Abschnitte}\index{Abschnitt} (\texttt{\textbackslash section})
und diese wiederum in 
\textsl{Unterabschnitte}\index{Unterabschnitt} (\texttt{\textbackslash subsection}) 
und so weiter~\cite{Kopka2000}. 

Bei den 
Dokumentklassen \verb!article! und \verb!proc! 
beginnt die Gliederung erst mit 
\textsl{Abschnitten} (\texttt{\textbackslash section}). 
\textsl{Kapitel} (\texttt{\textbackslash chapter}) stehen hier 
nicht zur Verfügung. 

Die Überschrift wird den Gliederungsbefehlen aus Tabelle~\ref{Tabelle_Gliederungsebenen}  in geschweiften Klammern übergeben. Zudem kann als optionales Argument in eckigen Klammern eine Kurzform der Überschrift definiert werden. Diese wird anstatt der langen Überschrift im Inhaltsverzeichnis und in der Kopfzeile verwendet.


\fbox{\texttt{\textbackslash}\textsl{Gliederungsbefehl}\texttt{[}\textsl{Kurzform}\texttt{]}\{\textsl{Überschrift}\texttt{\}}}

Bei den Dokumentklassen \verb!book! 
und \verb!report! erhält die
Gliederungsüberschrift eines Kapitels 
(\texttt{\textbackslash chapter}) eine einstellige 
Nummer, die eines \textsl{Abschnitts} 
(\texttt{\textbackslash section}) eine zweistellige 
und die Überschrift eines \textsl{Unterabschnitts} 
(\texttt{\textbackslash subsection}) eine dreistellige Nummer. 
Gliederungsbefehle, die in der Hierarchie 
tiefer als \textsl{Unterabschnitte} stehen, 
erhalten keine Nummer. Wenn die Nummer einer 
Gliederungsüberschrift mehr als eine Ziffer 
enthält, dann werden die Ziffern von \LaTeX\ jeweils mit 
einem Punkt voneinander getrennt.

Bei den Dokumentklassen \verb!article! 
und \verb!proc! erhalten die 
\textsl{Abschnitte} (\texttt{\textbackslash section})
eine einstellige und die\index{Unterabschnitt} 
\textsl{Unterabschnitte} (\texttt{\textbackslash subsection}) 
eine zweistellige Nummer.

Der Gliederungsbefehl \texttt{\textbackslash part} hat eine Sonderposition 
unter den Gliederungsbefehlen inne. Er beeinflusst nicht die Nummerierung 
der anderen Gliederungsbefehle.

Wird mit dem Befehl \texttt{\textbackslash chapter} ein neues Kapitel begonnen,
beendet \LaTeX\ automatisch die
aktuelle Seite und beginnt eine neue. 
Diese enthält ganz oben im Textfeld eine Zeile \glqq\textbf{Chapter \textsl{x}}\grqq. 
Das \textbf{\textsl{x}} 
steht für die Nummer des Kapitels.
Direkt darunter befindet sich die
Überschrift des Kapitels. Ist in einem Dokument das Erweiterungspaket
\verb!german.sty! oder alternativ \verb!ngerman.sty! eingebunden, setzt der \LaTeX-Compiler anstatt 
\textbf{Chapter \textsl{x}} das deutsche 
\textbf{Kapitel \textsl{x}}. 


\begin{table}[htb]
\centering
\caption{Auswirkungen von \texttt{(n)german.sty} auf die Gliederungselemente}
\label{Tabelle_Deutsche_Gliederungsebenen}       % Give a unique label
\begin{tabular}{lll}
\hline
Befehl & Standard & mit \texttt{(n)german.sty} \\
\hline
\texttt{\textbackslash prefacename} & Preface & Vorwort \\
\texttt{\textbackslash refname} & References & Literatur \\
\texttt{\textbackslash abstractname} & Abstract & Zusammenfassung \\
\texttt{\textbackslash bibname} & Bibliography & Literaturverzeichnis \\
\texttt{\textbackslash chaptername} & Chapter & Kapitel \\
\texttt{\textbackslash appendixname} & Appendix & Anhang \\
\texttt{\textbackslash contentsname} & Contents & Inhaltsverzeichnis \\
\texttt{\textbackslash listfigurename} & List of Figures & Abbildungsverzeichnis\\
\texttt{\textbackslash listtablename} & List of Tables & Tabellenverzeichnis \\
\texttt{\textbackslash indexname} & Index & Index \\
\texttt{\textbackslash figurename} & Figure & Abbildung \\
\texttt{\textbackslash tablename} & Table & Tabelle \\
\texttt{\textbackslash partname} & Part & Teil \\
\texttt{\textbackslash pagename} & Page & Seite \\
\hline
\end{tabular}
\index[cmd]{\texttt{\textbackslash prefacename}}
\index[cmd]{\texttt{\textbackslash refname}}
\index[cmd]{\texttt{\textbackslash abstractname}}
\index[cmd]{\texttt{\textbackslash bibname}}
\index[cmd]{\texttt{\textbackslash chaptername}}
\index[cmd]{\texttt{\textbackslash appendixname}}
\index[cmd]{\texttt{\textbackslash contentsname}}
\index[cmd]{\texttt{\textbackslash listfigurename}}
\index[cmd]{\texttt{\textbackslash listtablename}}
\index[cmd]{\texttt{\textbackslash indexname}}
\index[cmd]{\texttt{\textbackslash figurename}}
\index[cmd]{\texttt{\textbackslash tablename}}
\index[cmd]{\texttt{\textbackslash partname}}
\index[cmd]{\texttt{\textbackslash pagename}}
\end{table}

Tabelle~\ref{Tabelle_Deutsche_Gliederungsebenen} enthält eine Übersicht, wie das Erweiterungspaket \verb!(n)german.sty! 
die Bezeichnungen und Überschriften der 
Gliederungselemente verändert.

Ist bei einem Gliederungsbefehl weder
eine Nummerierung noch ein
Eintrag im Inhaltsverzeichnis gewollt, 
muss ein Stern $\ast$ zwischen 
Gliederungsbefehl und dem Argument mit der Überschrift stehen.

\fbox{\texttt{\textbackslash}\textsl{Gliederungsbefehl}\texttt{$\ast$\{}\textsl{Überschrift}\}}

Die Angabe einer verkürzten Überschrift in eckigen Klammern ist hierbei nicht möglich. Es würde auch keinen Sinn machen, hier eine verkürzte Überschrift anzugeben, da diese eh nie abgedruckt würde.

Bei jedem Aufruf
eines Gliederungsbefehls, 
wird der dazugehörige Zähler um eins 
erhöht. Wird ein Gliederungsbefehl 
der $\ast$-Form aufrufen, wird der
Zähler nicht verändert.

\subsection{Tiefe der Nummerierung bei der Gliederung ändern}

Bei der Dokumentklasse \verb!article! ist die Tiefe, 
bis zu der Gliederungsbefehle
durchnummeriert werden können, gleich drei, und bei den Dokumentklassen 
\verb!book! und \verb!report! ist die Schranke gleich zwei. 

Das bedeutet, dass \verb!book! und \verb!report! 
standardmäßig bis
einschließlich \texttt{\textbackslash subsection} und \verb!article! 
bis einschließlich \verb!\subsubsection! durchnummerieren.

Die Schranke realisiert unter \LaTeX\ der Zähler \verb!secnumdepth!\index[cmd]{\texttt{\textbackslash secnumdepth}} 
und eine Änderung seines Werts ermöglicht der Befehl \verb|\setcounter|.


\fbox{\texttt{\textbackslash setcounter\{secnumdepth\}\{}\textsl{Zahl}\texttt{\}}}


Würde beispielsweise in einem Dokument mit der 
Dokumentklasse \verb!book! der Wert \verb!secnumdepth! auf \verb!4! gesetzt, 
würde der \LaTeX-Compiler die Gliederungsbefehle bis einschließlich \verb!\paragraph!
durchnummerieren. Mögliche Werte für \verb!secnumdepth! sind bei den 
Dokumentklassen \verb!book! und 
\verb!report! \verb!-1,0,1,...,5!.
Bei der Dokumentklasse \verb!article! 
ist der Wertebereich \verb!0,1,...,5!.


\subsection{Anfangswert der Nummerierung ändern}

Die Nummerierung jeder Ebene der Gliederungshierarchie beginnt mit 
dem Wert \verb!1!. Soll der Zähler eines bestimmten 
Gliederungsbefehls nicht mit \verb!1! beginnen, muss dem Zähler mit dem 
Befehl \verb!\setcounter!\index[cmd]{\texttt{\textbackslash setcounter}} ein anderer Wert zugewiesen werden.

\fbox{\texttt{\textbackslash setcounter\{}\textsl{Gliederungsname}\texttt{\}\{}\textsl{Zahl}\texttt{\}}}

Dieser Befehl legt den Startwert des Zählers des 
angegebenen Gliederungsbefehls fest. \textsl{Gliederungsname} steht in der Syntax
für einen Gliederungsbefehl, allerdings ohne 
den üblichen Backslash \verb!\! am Anfang des Befehls.

Der Befehl \verb!\setcounter{section}{5}! zum Beispiel setzt den Zähler von
\texttt{\textbackslash section} auf den Wert \verb!5!. Beim nächsten Aufruf von 
\texttt{\textbackslash section} wird der Zähler um eins erhöht, und damit bekommt der nächste
Abschnitt (\texttt{\textbackslash section}) die Nummer \verb!6! und 
der folgende Abschnitt die Nummer \verb!7! und so weiter. 

Ein Anwendungsbeispiel, wo die manuelle Definition des Zählers eines Gliederungsbefehls hilfreich ist, sind große Dokumente, die in mehrere unabhängige \LaTeX-Dateien unterteilt sind, und die zusammen wie ein großes Dokument aussehen sollen.





\subsection{Zusätzliche Untergliederung bei Büchern}

Die Dokumentklasse \verb!book! ermöglicht noch eine weitere Möglichkeit
der Untergliederung. Es handelt sich hierbei um eine Untergliederung in einen Buchvorspann (u.a. Vorwort), Buchhauptteil (Kapitel und Anhänge) und Buchnachspann (u.a.  Literaturverzeichnis und Index). Den Beginn jedes dieser drei Teile markieren die Befehle 
\verb!\frontmatter!, \verb!\mainmatter! und \verb!\backmatter!.
\index{Buchvorspann}
\index{Buchhauptteil}
\index{Buchnachspann}

Der Dokumentteil, der nach dem Befehl \verb!\frontmatter!\index[cmd]{\texttt{\textbackslash frontmatter}} 
folgt, erhält von \LaTeX\ kleine römische Zahlen für die Seitennummerierung.
Gliederungsbefehle in diesem Teil erhalten keine Nummerierung. 

\fbox{\texttt{\textbackslash frontmatter}}

Im Hauptteil des Buches, der nach dem Befehl \verb!\mainmatter!\index[cmd]{\texttt{\textbackslash backmatter}} 
folgt, erfolgt die Seitennummerierung
mit arabischen Ziffern. Der Zähler für die Seitenzahlen wird automatisch
zurückgesetzt. Das heißt, dass die erste Seite des Hauptteils eine arabische 
\verb!1! als Seitenzahl erhält. Gliederungsbefehle in diesem Teil erhalten eine
fortlaufende Nummerierung. 

\fbox{\texttt{\textbackslash mainmatter}}

Im Nachspann des Buches, der nach dem Befehl \verb!\backmatter!\index[cmd]{\texttt{\textbackslash backmatter}} 
folgt, ist die Gliederungsnummerierung wie im Buchvorspann
abgeschaltet. Die Seitenzahlen werden aber wie im Hauptteil
in arabischen Ziffern geschrieben und der Zähler wird nicht zurückgesetzt.

\fbox{\texttt{\textbackslash backmatter}}


\subsection{Anhang}
\index{Anhang}

Zur Realisierung des Anhangs existiert der 
Befehl \verb!\appendix!\index[cmd]{\texttt{\textbackslash appendix}}.

\fbox{\texttt{\textbackslash appendix}}

An der Stelle im Quelltext, wo dieser Befehl aufgerufen wird, setzt \LaTeX\ 
bei den Dokumentklassen \verb!book! und \verb!report! den Zähler der Kapitel (\verb!chapter!) auf null zurück. 

Bei der Dokumentklasse \verb!article!, bei der ja keine Kapitel 
(\texttt{\textbackslash chapter}) existieren, wird der Zähler der Abschnitte
(\verb!chapter!) auf null zurückgesetzt. 

Im Anhang nummeriert der \LaTeX.Compiler die Kapitel, bzw. bei \verb!article! die Abschnitte, nicht mehr mit arabischen Ziffern, sondern mit großen Buchstaben. Dabei steht
\verb!A! für \verb!1!  und \verb!Z! für \verb!26!. Außerdem wird das Wort
\textbf{Chapter} bzw. \textbf{Kapitel} (bei \texttt{(n)german.sty}) durch das Wort 
\textbf{Appendix} bzw. \textbf{Anhang} ersetzt.

Alle in der Hierarchie darunter gelegenen Gliederungsbefehle werden ganz normal
mit arabischen Ziffern durchnummeriert. So könnte ein Unterabschnitt im Anhang
eines Dokuments mit der Dokumentklasse \verb!book! beispielsweise die
Nummer \textbf{D.5.3} tragen.


\section{Inhaltsverzeichnis}
\index{Inhaltsverzeichnis}

Das Inhaltsverzeichnis (\textsl{Table of Contents}) enthält die
Abschnittsüberschriften mit den zugehörigen Seitennummern, die der \LaTeX-Compiler beim Durchlauf in der 
\verb!.toc!-Datei ablegt. Diese Datei wird, wenn der 
Befehl \verb!\tableofcontents!\index[cmd]{\texttt{\textbackslash tableofcontents}}  
aufgerufen wird und vorausgesetzt sie existiert, bei der Übersetzung eines
Dokuments eingelesen und als Inhaltsverzeichnis formatiert 
auf einer neuen Seite ausgegeben. 

\fbox{\texttt{\textbackslash tableofcontents}}

Wegen des Konzepts von \LaTeX\ schlagen sich Änderungen in der Struktur
(Gliederung) des Dokuments sich erst nach zwei Durchläufen des \LaTeX-Compilers im
Inhaltsverzeichnis nieder.

Der Grund dafür ist, dass die (aktuellsten) Informationen über das Inhaltsverzeichnis erst nach einer
kompletten Abarbeitung des Dokuments vorliegen können. Da aber das
Inhaltsverzeichnis normalerweise am Anfang des Dokuments liegt und nicht am
Ende, muss bei Änderungen am Inhaltsverzeichnis der \LaTeX-Compiler zwei mal das Dokument durchlaufen. 
Beim ersten Durchlauf durch das Dokument weist der
Befehl \verb!\tableofcontents! den \LaTeX-Compiler an, eine Datei mit dem Stammnamen des Dokuments
und der Endung \verb!.toc! anzulegen. In diese Datei schreibt der Compiler die
verwendete Sprache, Gliederungsbefehle und ihre zugehörigen Überschriften
sowie Seitenzahlen. Diese Datei wird während jeder Abarbeitung des Dokuments befüllt
bzw. aktualisiert. 

Typische Einträge in der \verb!.toc!-Datei sehen wie folgt aus:

\fbox{\texttt{\textbackslash contentsline\{}\textsl{Gliederungsname}\texttt{\}\{}\textsl{Überschrift}\texttt{\}\{}\textsl{Seitennummer}\texttt{\}}}

\textsl{Gliederungsname} steht für den Gliederungsbefehl, der
verwendet wurde, also zum Beispiel \texttt{\textbackslash chapter}, \texttt{\textbackslash section} oder \texttt{\textbackslash subsection}.
\textsl{Überschrift} steht für die Gliederungsüberschrift
bzw. für die optionale Kurzform der Überschrift, die in eckigen Klammern
zwischen Gliederungsbefehl und Gliederungsüberschrift angeben sein kann. 
Die \textsl{Seitennummer}, der letzte Eintrag einer jeden Zeile, steht
für die zugehörige Seitennummer im Dokument.

\subsection{Manuell Einträge in das Inhaltsverzeichnis einfügen}
\index{Inhaltsverzeichnis!Eintrag}


Einige Situationen erfordern es, eigene Eintragungen in das Inhaltsverzeichnis 
zu setzen. Denkbare Szenarien sind der 
Einsatz von Gliederungsbefehlen der 
$\ast$-Form oder wenn ein Gliederungsbefehl (z.B. Literaturverzeichnis oder Index) verwendet wird, der keinen automatischen Eintrag im Inhaltsverzeichnis erzeugt. 


Der Befehl \verb!\addcontentsline!\index[cmd]{\texttt{\textbackslash addcontentsline}} ermöglicht das manuelle Einfügen von Einträgen in das Inhaltsverzeichnis.

\fbox{\texttt{\textbackslash addcontentsline\{toc\}\{}\textsl{Gliederungsname}\texttt{\}\{}\textsl{Text}\texttt{\}}}


\textsl{Gliederungsname} ist der 
Name des gewünschten Gliederungsbefehls ohne den für einen Befehl 
typischen, vorgestellten Backslash \verb!\!. 
Dieser \textsl{Gliederungsname} ist 
für die Einrückung des Eintrags im
Inhaltsverzeichnis verantwortlich.
Bei der Dokumentklasse \verb!book! ist ein \verb!chapter!-Eintrag überhaupt 
nicht eingerückt. Ein \verb!section!-Eintrag 
hingegen schon, und ein 
\verb!subsection!-Eintrag noch mehr. 


\textsl{Text} ist der neue Eintrag im
Inhaltsverzeichnis, der vor der Seitennummer 
stehen soll.


\section{Abbildungs- und Tabellenverzeichnis}
\index{Abbildungsverzeichnis}
\index{Tabellenverzeichnis}

Bei umfangreichen Dokumenten sind ein Abbildungsverzeichnis und ein Tabellenverzeichnis häufig sinnvoll. 

Der Befehl \verb!\listoffigures!\index[cmd]{\texttt{\textbackslash listoffigures}} 
fügt an der Stelle im Quelltext, an der er aufgerufen wird, auf einer neuen Seite ein Abbildungsverzeichnis ein. 

\fbox{\texttt{\textbackslash listoffigures}} 

Das Tabellenverzeichnis fügt der Befehl \verb!\listoftables!\index[cmd]{\texttt{\textbackslash listoftables}} 
an der aufgerufenen Stelle auf einer neuen Seite ein.

\fbox{\texttt{\textbackslash listoftables}} 

Eintragungen in diese Verzeichnisse erfolgen automatisch immer dann, wenn der Befehl
\verb!\caption! aufgerufen wird. Dieser Befehl wird bei den Umgebungen \verb!figure! und \verb!table! angewendet, um die Bildunter- bzw. Tabellenunterschrift zu setzen.


\section{Querverweise}
\index{Querverweis}

Mit dem Befehl \verb!\label! werden in Dokumenten 
unsichtbare Markierungen\index{Markierung} eingefügt, auf die mit den Befehlen 
\verb!\ref! und \verb!\pageref! referenziert werden kann.

\begin{boxedminipage}{\textwidth}
\texttt{\textbackslash label\{}\textsl{Markierung}\texttt{\}} \\
\texttt{\textbackslash pageref\{}\textsl{Markierung}\texttt{\}} \\
\texttt{\textbackslash ref\{}\textsl{Markierung}\texttt{\}}
\end{boxedminipage}

Die \textsl{Markierung}, die dem Befehl \verb!\label!
als Argument übergeben wird, kann 
eine fast beliebige Folge von Buchstaben,
Zahlen und Zeichen aller Art sein. Auf die Verwendung der Sonderzeichen
aus Tabelle~\ref{Tabelle_Sonderzeichen} sollte hier in jedem Fall verzichtet werden.


Mit dem Befehl \verb!\pageref!
ist es möglich, auf eine Seitenzahl zu referenzieren.
An der Position, an der der Befehl \verb!\pageref! auftritt, 
erscheint die Seitennummer, an der sich die 
\textsl{Markierung} befindet, auf die die \verb!\pageref! referenziert.

Ein weiterer Befehl, um Querverweise zu erzeugen, 
ist der Befehl \verb!\ref!. Mit diesem ist es möglich auf eine
Gleichungs-, Gliederungs-, Tabellen- oder Bildnummer zu referenzieren.
Was konkret referenziert wird, hängt davon ab, ob sich der Befehl \verb!\label! in einer 
mathematischen Umgebung wie zum Beispiel
\verb!math! oder \verb!displaymath! befindet, oder in nach einem Gliederungsbefehl (z.B. 
\texttt{\textbackslash chapter}, \texttt{\textbackslash section} oder \texttt{\textbackslash subsection}), oder in einer Tabellenumgebung (z.B. \verb!table!) oder
in der Umgebung \verb!figure! zum Satz einer Abbildung, etc. Weitere sinnvolle Anwendungen, wie zum Beispiel die 
Referenzierung auf Fußnoten sind denkbar.


\section{Index}
% In das Stichwortverzeichnis schreibe ich hier nicht Index, weil das Wort schon in der Mathematik vorkommt.
\index{Indexregister}
\index{Stichwortverzeichnis}

Die Erzeugung eines
Indexregisters (Stichwortverzeichnis)
geschieht weitgehend automatisiert.
Hierfür muss das Erweiterungspaket \verb!makeidx! mit dem Befehl \verb!\usepackage{makeidx}! in der Präambel der \verb!.tex!-Quelldatei eingebunden sein.
Zudem müssen im Quelltext des Dokuments an den 
entsprechenden Stellen die 
Einträge für das Indexregister mit dem Befehl \verb!\index! definiert sein. 

\begin{boxedminipage}{\textwidth}
\texttt{\textbackslash index\{}\textsl{Indexeintrag}\texttt{\}} 
\end{boxedminipage}

Zusätzlich muss in der Präambel des Dokuments der Befehl \verb!\makeindex!\index[cmd]{\texttt{\textbackslash makeindex}} stehen
und an der Stelle im Dokument, an der das Indexregister erscheinen soll, 
ist der Befehl \verb!\printindex!\index[cmd]{\texttt{\textbackslash printindex}} nötig.


\begin{boxedminipage}{\textwidth}
\texttt{\textbackslash makeindex} \\
\texttt{\textbackslash printindex} 
\end{boxedminipage}

Die Schritte zur Erstellung des Stichwortverzeichnisses 
sind wie folgt:

Nachdem die Quelldatei vom \LaTeX-Compiler abgearbeitet wurde\dots

\textbf{1)} \verb!pdflatex dateiname.tex!

erzeugt der \LaTeX-Compiler eine Datei 
\verb!dateiname.idx!\index{\texttt{idx}-Datei}, die 
die Einträge für das Indexregister 
enthält. Anschließend muss ein Durchlauf des Programms
\verb!makeindex! folgen:

\textbf{2)} \verb!makeindex dateiname.idx!

Das Programm \verb!makeindex! legt eine Datei \verb!dateiname.ind!\index{\texttt{ind}-Datei} an, 
die ebenfalls alle Einträge für das Indexregister enthält,
aber im Gegensatz zu \verb!dateiname.idx! sind die Einträge hier
alphabetisch sortiert und die Seitenzahlen beigefügt.

Ein abschließender Durchlauf des \LaTeX-Compilers sorgt dafür, dass das alphabetisch 
korrekt sortierte Indexregister an der Stelle im Dokument eingefügt 
wird, wo der Befehl \verb!\printindex! eingefügt ist.

\textbf{3)} \verb!pdflatex dateiname.tex!
