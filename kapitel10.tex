\chapter{Sonderschriften}

Standardmäßig verwendet \LaTeX\ die Schrift Computer Modern. Diese verleiht 
Dokumenten einen gewissen \glqq\LaTeX-Look\grqq. 
Die Verwendung anderer Schriften, auch ungewöhnlicher Schriften, ist auf verschiedene
Art und Weise möglich.

Eine hilfreiche Übersicht über freie Schriften und Beschreibungen, wie diese in eigenen Dokumenten verwendet
werden können, bietet der \LaTeX\ Font Catalogue~\cite{LaTeXFontCatalogue}.

\section{Altdeutsche Schriften}
\index{Altdeutsche Schrift}

Eine Möglichkeit, um altdeutsche Schriften zu verwenden, ist die Nutzung des 
Erweiterungspakets \verb!yfonts!~\cite{TypesettingOldGermanFonts_Dokumentation}. Nachdem es mit dem Befehl \verb!\usepackage{yfonts}! in der Präambel der Quelldatei
eingebunden ist, stehen die Befehle 
\verb!\textgoth! \index[cmd]{\texttt{\textbackslash textgoth}}
für \textgoth{Gotische Schrift}, 
\verb!\textswab! \index[cmd]{\texttt{\textbackslash textswab}}
für \textswab{Schwabacher Schrift}
und \verb!\textfrak! \index[cmd]{\texttt{\textbackslash textfrak}}
für \textfrak{Fraktur} zur Verfügung.

\begin{boxedminipage}{\textwidth}
\texttt{\textbackslash textgoth\{}\textsl{Text}\texttt{\}} \\
\texttt{\textbackslash textswab\{}\textsl{Text}\texttt{\}} \\
\texttt{\textbackslash textfrak\{}\textsl{Text}\texttt{\}}
\end{boxedminipage}

\section{Barcodes}
\index{Barcode}

Es existieren mehrere Erweiterungspakete, um Barcodes komfortabel zu setzen.
Ein Beispiel ist das Erweiterungspaket \verb!makebarcode!~\cite{Barcodes_Dokumentation}, das den Befehl \verb!\barcode!\index[cmd]{\texttt{\textbackslash barcode}} definiert.

\begin{boxedminipage}{\textwidth}
\texttt{\textbackslash barcode\{}\textsl{Text}\texttt{\}} 
\end{boxedminipage}

Der folgende Befehl in der Präambel der Quelldatei bindet das Erweiterungspaket ein und definiert die Kodierung Code 39, 
eine Zeichenhöhe (Parameter \verb!H!) von 1\,cm, das dünne Linien 0,5\,mm breit (Parameter \verb!X!) sind und das dicke Linien 2,5 mal so breit sind wie dünne Linien (Parameter \verb!ratio!).

\begin{boxedminipage}{\textwidth}
\verb!\usepackage[code=Code39,X=.5mm,ratio=2.5,H=1cm]{makebarcode}!
\end{boxedminipage}

In der Praxis sehen einzelne Zeichen dann wie in Tabelle~\ref{Tabelle_Barcodes} aus.

\begin{table}[h!tb]
\centering
\begin{longtable}{ccccc}
\caption{Einzelne Zeichen, als Barcodes gesetzt}
\label{Tabelle_Barcodes}       % Give a unique label
\endfirsthead
\endhead
0 & 1 & 2 & 3 & 4 \\ 
\barcode{0} & \barcode{1} & \barcode{2} & \barcode{3} & \barcode{4} \\
5 & 6 & 7 & 8 & 9 \\
\barcode{5} & \barcode{6} & \barcode{7} & \barcode{8} & \barcode{9} \\
+ & - & / & . & A \\
\barcode{+} & \barcode{-} & \barcode{/} & \barcode{.} & \barcode{A} \\
B & C & D & E & F \\
\barcode{B} & \barcode{C} & \barcode{D} & \barcode{E} & \barcode{F} \\
G & H & I & J & K \\
\barcode{G} & \barcode{H} & \barcode{I} & \barcode{J} & \barcode{K} \\
L & M & N & O & P \\
\barcode{L} & \barcode{M} & \barcode{N} & \barcode{O} & \barcode{P} \\
Q & R & S & T & U \\
\barcode{Q} & \barcode{R} & \barcode{S} & \barcode{T} & \barcode{U} \\
V & W & X & Y & Z \\
\barcode{V} & \barcode{W} & \barcode{X} & \barcode{Y} & \barcode{Z} \\
\end{longtable}
\end{table}


\section{QR-Codes}
\label{Abschnitt_QR-Code}
\index{QR-Code}

Bei QR-Codes\footnote{Das QR steht für Quick Response} handelt es sich um zweidimensionale Codes für Zeichenketten.
Ein QR-Code besteht aus einer quadratischen Matrix aus schwarzen und weißen Quadraten, die die kodierten Daten binär darstellen. 
Der maximale Informationsgehalt eines QR-Codes (177x177 Elemente) beträgt ca. 3\,kB).~\cite{QRCode_Wikipedia}
Mobiltelefone verfügen in der Regel über eine Software, die in der 
Lage ist, QR-Codes zu erkennen und zu dekodieren. Eine typische Anwendung ist es, Internetadressen als
QR-Codes bereitzustellen, um diese einfach und ohne manuelle Eingabe an ein mobiles Gerät zu übertragen.

Zum Erstellen von QR-Codes ist es unter anderem möglich, das Erweiterungspaket \verb!qrcode! zu nutzen.
Dieses Paket bietet zahlreiche Möglichkeiten, Informationen als QR-Codes zu kodieren, und es definiert hierzu den 
Befehl \verb!\qrcode!\index[cmd]{\texttt{\textbackslash qrcode}}.




\begin{boxedminipage}{\textwidth}
\texttt{\textbackslash qrcode\{}\textsl{Optionen}\texttt{\}\{}\textsl{Text}\texttt{\}}
\end{boxedminipage}


Von den möglichen Optionen des Befehls ist an dieser Stelle nur \verb!height! vorgestellt. Hier ist die Höhe (und Breite) des zu 
erzeugenden QR-Codes inklusive einer Maßeinheit 
(siehe Abschnitt~\ref{sec:Massangaben}) definiert. Um beispielsweise einen QR-Code mit der URL zu diesem Werk zu erzeugen, der 3\,cm hoch und breit ist, 
genügt der folgende Befehl: 

\begin{lstlisting}[label=QRcode, style=customlatex]
\qrcode[height=3cm]{https://github.com/christianbaun/einstieginlatex}
\end{lstlisting}

Das Ergebnis sieht wie folgt aus:

\qrcode[height=3cm]{https://github.com/christianbaun/einstieginlatex}

\section{Besondere Zeichen}

Ein Zeichensatz, der statt Buchstaben, 
Ziffern und Satzzeichen mit 
Sonderzeichen aller Art gefüllt ist, ist \textsl{ZapfDingbats}\index{ZapfDingbats} (siehe Tabelle~\ref{Tabelle_Font_ZapfDingbats}).
Dessen Verwendung ist mit dem Erweiterungspaket \verb!pifont.sty! 
besonders einfach. 

\begin{longtable}{c|p{.9cm}|p{.9cm}|p{.9cm}|p{.9cm}|p{.9cm}|p{.9cm}|p{.9cm}|p{.9cm}}
\caption{Zeichen aus dem Font ZapfDingbats}
\label{Tabelle_Font_ZapfDingbats}       % Give a unique label
\endfirsthead
\endhead
& \multicolumn{1}{c|}{\textquotesingle 0} & \multicolumn{1}{c|}{\textquotesingle 1} & \multicolumn{1}{c|}{\textquotesingle 2} 
& \multicolumn{1}{c|}{\textquotesingle 3} & \multicolumn{1}{c|}{\textquotesingle 4} & \multicolumn{1}{c|}{\textquotesingle 5}
& \multicolumn{1}{c|}{\textquotesingle 6} & \multicolumn{1}{c}{\textquotesingle 7} \\
\hline
\textquotesingle 04x & & 
{\ding{33}}\hfill\tiny{33} & 
{\ding{34}}\hfill\tiny{34} & 
{\ding{35}}\hfill\tiny{35} & 
{\ding{36}}\hfill\tiny{36} & 
{\ding{37}}\hfill\tiny{37} & 
{\ding{38}}\hfill\tiny{38} & 
{\ding{39}}\hfill\tiny{39}  \\
\hline
\textquotesingle 05x & 
{\ding{40}}\hfill\tiny{40} & 
{\ding{41}}\hfill\tiny{41} & 
{\ding{42}}\hfill\tiny{42} & 
{\ding{43}}\hfill\tiny{43} & 
{\ding{44}}\hfill\tiny{44} & 
{\ding{45}}\hfill\tiny{45} & 
{\ding{46}}\hfill\tiny{46} & 
{\ding{47}}\hfill\tiny{47}  \\ 
\hline
\textquotesingle 06x & 
{\ding{48}}\hfill\tiny{48} & 
{\ding{49}}\hfill\tiny{49} & 
{\ding{50}}\hfill\tiny{50} & 
{\ding{51}}\hfill\tiny{51} & 
{\ding{52}}\hfill\tiny{52} & 
{\ding{53}}\hfill\tiny{53} & 
{\ding{54}}\hfill\tiny{54} & 
{\ding{55}}\hfill\tiny{55}  \\
\hline
\textquotesingle 07x & 
{\ding{56}}\hfill\tiny{56} & 
{\ding{57}}\hfill\tiny{57} & 
{\ding{58}}\hfill\tiny{58} & 
{\ding{59}}\hfill\tiny{59} & 
{\ding{60}}\hfill\tiny{60} & 
{\ding{61}}\hfill\tiny{61} & 
{\ding{62}}\hfill\tiny{62} & 
{\ding{63}}\hfill\tiny{63}  \\
\hline
\textquotesingle 10x &
{\ding{64}}\hfill\tiny{64} & 
{\ding{65}}\hfill\tiny{65} & 
{\ding{66}}\hfill\tiny{66} & 
{\ding{67}}\hfill\tiny{67} & 
{\ding{68}}\hfill\tiny{68} & 
{\ding{69}}\hfill\tiny{69} & 
{\ding{70}}\hfill\tiny{70} & 
{\ding{71}}\hfill\tiny{71} \\
\hline
\textquotesingle 11x & 
{\ding{72}}\hfill\tiny{72} & 
{\ding{73}}\hfill\tiny{73} & 
{\ding{74}}\hfill\tiny{74} & 
{\ding{75}}\hfill\tiny{75} & 
{\ding{76}}\hfill\tiny{76} & 
{\ding{77}}\hfill\tiny{77} & 
{\ding{78}}\hfill\tiny{78} & 
{\ding{79}}\hfill\tiny{79}  \\
\hline
\textquotesingle 12x & 
{\ding{80}}\hfill\tiny{80} & 
{\ding{81}}\hfill\tiny{81} & 
{\ding{82}}\hfill\tiny{82} & 
{\ding{83}}\hfill\tiny{83} & 
{\ding{84}}\hfill\tiny{84} & 
{\ding{85}}\hfill\tiny{85} & 
{\ding{86}}\hfill\tiny{86} & 
{\ding{87}}\hfill\tiny{87}  \\
\hline
\textquotesingle 13x &
{\ding{88}}\hfill\tiny{88} & 
{\ding{89}}\hfill\tiny{89} & 
{\ding{90}}\hfill\tiny{90} & 
{\ding{91}}\hfill\tiny{91} & 
{\ding{92}}\hfill\tiny{92} & 
{\ding{93}}\hfill\tiny{93} & 
{\ding{94}}\hfill\tiny{94} & 
{\ding{95}}\hfill\tiny{95} \\
\hline
\textquotesingle 14x & 
{\ding{96}}\hfill\tiny{96} & 
{\ding{97}}\hfill\tiny{97} & 
{\ding{98}}\hfill\tiny{98} & 
{\ding{99}}\hfill\tiny{99} & 
{\ding{100}}\hfill\tiny{100} & 
{\ding{101}}\hfill\tiny{101} & 
{\ding{102}}\hfill\tiny{102} & 
{\ding{103}}\hfill\tiny{103}  \\
\hline
\textquotesingle 15x & 
{\ding{104}}\hfill\tiny{104} & 
{\ding{105}}\hfill\tiny{105} & 
{\ding{106}}\hfill\tiny{106} & 
{\ding{107}}\hfill\tiny{107} & 
{\ding{108}}\hfill\tiny{108} & 
{\ding{109}}\hfill\tiny{109} & 
{\ding{110}}\hfill\tiny{110} & 
{\ding{111}}\hfill\tiny{111}  \\
\hline
\textquotesingle 16x & 
{\ding{112}}\hfill\tiny{112} & 
{\ding{113}}\hfill\tiny{113} & 
{\ding{114}}\hfill\tiny{114} & 
{\ding{115}}\hfill\tiny{115} & 
{\ding{116}}\hfill\tiny{116} & 
{\ding{117}}\hfill\tiny{117} & 
{\ding{118}}\hfill\tiny{118} & 
{\ding{119}}\hfill\tiny{119}  \\
\hline
\textquotesingle 17x & 
{\ding{120}}\hfill\tiny{120} & 
{\ding{121}}\hfill\tiny{121} & 
{\ding{122}}\hfill\tiny{122} & 
{\ding{123}}\hfill\tiny{123} & 
{\ding{124}}\hfill\tiny{124} & 
{\ding{125}}\hfill\tiny{125} & 
{\ding{126}}\hfill\tiny{126} &  \\
\hline
\textquotesingle 26x & & 
{\ding{161}}\hfill\tiny{161} & 
{\ding{162}}\hfill\tiny{162} & 
{\ding{163}}\hfill\tiny{163} & 
{\ding{164}}\hfill\tiny{164} & 
{\ding{165}}\hfill\tiny{165} & 
{\ding{166}}\hfill\tiny{166} & 
{\ding{167}}\hfill\tiny{167}  \\
\hline
\textquotesingle 27x & 
{\ding{168}}\hfill\tiny{168} & 
{\ding{169}}\hfill\tiny{169} & 
{\ding{170}}\hfill\tiny{170} & 
{\ding{171}}\hfill\tiny{171} & 
{\ding{172}}\hfill\tiny{172} & 
{\ding{173}}\hfill\tiny{173} & 
{\ding{174}}\hfill\tiny{174} & 
{\ding{175}}\hfill\tiny{175} \\
\hline
\textquotesingle 28x & 
{\ding{176}}\hfill\tiny{176} & 
{\ding{177}}\hfill\tiny{177} & 
{\ding{178}}\hfill\tiny{178} & 
{\ding{179}}\hfill\tiny{179} & 
{\ding{180}}\hfill\tiny{180} & 
{\ding{181}}\hfill\tiny{181} & 
{\ding{182}}\hfill\tiny{182} & 
{\ding{183}}\hfill\tiny{183} \\
\hline
\textquotesingle 29x & 
{\ding{184}}\hfill\tiny{184} & 
{\ding{185}}\hfill\tiny{185} & 
{\ding{186}}\hfill\tiny{186} & 
{\ding{187}}\hfill\tiny{187} & 
{\ding{188}}\hfill\tiny{188} & 
{\ding{189}}\hfill\tiny{189} & 
{\ding{190}}\hfill\tiny{190} & 
{\ding{191}}\hfill\tiny{191} \\
\hline
\textquotesingle 30x & 
{\ding{192}}\hfill\tiny{192} & 
{\ding{193}}\hfill\tiny{193} & 
{\ding{194}}\hfill\tiny{194} & 
{\ding{195}}\hfill\tiny{195} & 
{\ding{196}}\hfill\tiny{196} & 
{\ding{197}}\hfill\tiny{197} & 
{\ding{198}}\hfill\tiny{198} & 
{\ding{199}}\hfill\tiny{199}  \\
\hline
\textquotesingle 31x & 
{\ding{200}}\hfill\tiny{200} & 
{\ding{201}}\hfill\tiny{201} & 
{\ding{202}}\hfill\tiny{202} & 
{\ding{203}}\hfill\tiny{203} & 
{\ding{204}}\hfill\tiny{204} & 
{\ding{205}}\hfill\tiny{205} & 
{\ding{206}}\hfill\tiny{206} & 
{\ding{207}}\hfill\tiny{207}  \\
\hline
\textquotesingle 32x & 
{\ding{208}}\hfill\tiny{208} & 
{\ding{209}}\hfill\tiny{209} & 
{\ding{210}}\hfill\tiny{210} & 
{\ding{211}}\hfill\tiny{211} & 
{\ding{212}}\hfill\tiny{212} & 
{\ding{213}}\hfill\tiny{213} & 
{\ding{214}}\hfill\tiny{214} & 
{\ding{215}}\hfill\tiny{215}  \\
\hline
\textquotesingle 33x & 
{\ding{216}}\hfill\tiny{216} & 
{\ding{217}}\hfill\tiny{217} & 
{\ding{218}}\hfill\tiny{218} & 
{\ding{219}}\hfill\tiny{219} & 
{\ding{220}}\hfill\tiny{220} & 
{\ding{221}}\hfill\tiny{221} & 
{\ding{222}}\hfill\tiny{222} & 
{\ding{223}}\hfill\tiny{223}  \\
\hline
\textquotesingle 34x & 
{\ding{224}}\hfill\tiny{224} & 
{\ding{225}}\hfill\tiny{225} & 
{\ding{226}}\hfill\tiny{226} & 
{\ding{227}}\hfill\tiny{227} & 
{\ding{228}}\hfill\tiny{228} &
{\ding{229}}\hfill\tiny{229} &
{\ding{230}}\hfill\tiny{230} &
{\ding{231}}\hfill\tiny{231}  \\
\hline
\textquotesingle 35x & 
{\ding{232}}\hfill\tiny{232} & 
{\ding{233}}\hfill\tiny{233} & 
{\ding{234}}\hfill\tiny{234} & 
{\ding{235}}\hfill\tiny{235} & 
{\ding{236}}\hfill\tiny{236} &
{\ding{237}}\hfill\tiny{237} &
{\ding{238}}\hfill\tiny{238} &
{\ding{239}}\hfill\tiny{239}  \\
\hline
\textquotesingle 36x & & 
{\ding{241}}\hfill\tiny{241} & 
{\ding{242}}\hfill\tiny{242} & 
{\ding{243}}\hfill\tiny{243} & 
{\ding{244}}\hfill\tiny{244} &
{\ding{245}}\hfill\tiny{245} &
{\ding{246}}\hfill\tiny{246} &
{\ding{247}}\hfill\tiny{247}  \\
\hline
\textquotesingle 37x & 
{\ding{248}}\hfill\tiny{248} & 
{\ding{249}}\hfill\tiny{249} & 
{\ding{250}}\hfill\tiny{250} & 
{\ding{251}}\hfill\tiny{251} & 
{\ding{252}}\hfill\tiny{252} &
{\ding{253}}\hfill\tiny{253} &
{\ding{254}}\hfill\tiny{254} &  \\
\end{longtable}

Das Einbinden des Erweiterungspakets in eigene Dokumente 
geschieht mit dem Befehl \verb!\usepackage{pifont}! in der 
Präambel des Dokuments.

Ein Befehl, den das Erweiterungspaket definiert, ist \verb!\ding!\index[cmd]{\texttt{ding}}, um einzelne Zeichen aus dem Font
\textsl{ZapfDingbats} auszugeben.

\begin{boxedminipage}{\textwidth}
\texttt{\textbackslash ding\{}\textsl{Zeichen}\texttt{\}} 
\end{boxedminipage}

Dem Befehl wird die Nummer (siehe Tabelle~\ref{Tabelle_Font_ZapfDingbats}) des Zeichens, das gesetzt werden soll, in geschweiften Klammern als Argument übergeben. Die Schneeflocke 
\ding{100} mit der Zeichennummer \verb!100! beispielsweise setzt der Befehl 
\verb!\ding{100}!. 

Alternativ kann auf die einzelnen Zeichen auch durch die Angabe der Kombination aus Zeilen- und Spaltennummer angeben werden. Die In diesem Fall wird ein einzelnes Anführungszeichen der Zeichennummer vorgestellt. Die bereits beschriebene Schneeflocke befindet sich in Zeile 14 und Spalte 4. Das heißt, sie kann auch mit diesem Befehl gesetzt werden: \verb!\ding{'144}!

Zwei weitere Befehle, die das Erweiterungspaket \verb!pifont! definiert,
sind \verb!\dingline! und \verb!\dingfill!.

\begin{boxedminipage}{\textwidth}
\texttt{\textbackslash dingline\{}\textsl{Zeichen}\texttt{\}} \\
\texttt{\textbackslash dingfill\{}\textsl{Zeichen}\texttt{\}} 
\end{boxedminipage}

Wird der Befehl \verb!\dingline!\index[cmd]{\texttt{dingline}} ausgeführt, beginnt der \LaTeX-Compiler eine neue Zeile, die mit dem als Argument übergeben  
Zeichen gefüllt wird. Die beidseitige Einrückung
entspricht 0,5\,Zoll (also ca. 1,3\,cm). Ein sinnvolles Anwendungsbeispiel für diesen Befehl sind Formulare, wo ein Teil des Blattes 
abgetrennt werden soll, beispielsweise so: 

\dingline{34}

Die Zeile mit den Scheren erzeugt der Befehl \verb!\dingline{34}!.

Im Gegensatz zu \verb!\dingline! wird bei \verb!\dingfill!\index[cmd]{\texttt{dingfill}} nur der 
in einer Zeile verfügbare Leerraum mit dem als Argument übergeben  
Zeichen gefüllt. So erzeugt der Befehl \verb!\dingfill{51}! hier: \dingfill{51}

Der Befehl \verb!\dingfill! füllt immer bis zum Zeilenende.
Es sei denn, es folgt noch ein Text nach dem Befehl. 
Der Befehl \verb!\dingfill{51}! gefolgt von der Zeichenkette \verb!Ende.! erzeugt folgendes Ergebnis: \dingfill{51} Ende. 

Außer den bereits vorgestellten Befehlen definiert das Erweiterungspaket \verb!pifont! auch die Umgebung
\verb!dinglist!\index[cmd]{\texttt{dinglist}}. Diese ermöglicht es Aufzählungen (siehe Abschnitt~\ref{Abschnitt_Aufzaehlungen})
ähnlich wie \verb!itemize! zu setzen.

\begin{boxedminipage}{\textwidth}
\texttt{\textbackslash begin\{dinglist\}\{}\textsl{Zeichen}\texttt{\}}  \enskip \dots\ \enskip \texttt{\textbackslash end\{dinglist\}} 
\end{boxedminipage}

Die einzelnen Listenpunkte beginnen wie gehabt mit dem Befehl \verb!\item!.

Bei der Umgebung \verb!dinglist! wird die Nummer (siehe Tabelle~\ref{Tabelle_Font_ZapfDingbats}) des Zeichens, das als Markierungszeichen für jeden Auflistungspunkt verwendet werden soll, in geschweiften Klammern als Argument übergeben.

\begin{minipage}[h]{0.5\textwidth}
\setlength{\parskip}{1em}
\frenchspacing
\begin{Verbatim}[frame=single]
\begin{dinglist}{52}
  \item Erster Listenpunkt
  \item Zweiter Listenpunkt
  \begin{dinglist}{234}
    \item Erster Unterpunkt
    \item Zweiter Unterpunkt
  \end{dinglist}
  \item Dritter Listenpunkt
\end{dinglist}
\end{Verbatim}
\end{minipage}
\hfill
\begin{minipage}[h]{0.48\textwidth}
\setlength{\parskip}{1em}
\frenchspacing
\begin{dinglist}{52}
  \item Erster Listenpunkt
  \item Zweiter Listenpunkt
  \begin{dinglist}{234}
    \item Erster Unterpunkt
    \item Zweiter Unterpunkt
  \end{dinglist}
  \item Dritter Listenpunkt
\end{dinglist}
\end{minipage}

Neben der Umgebung \verb!dinglist!, die ein Äquivalent zur Umgebung \verb!itemize! ist, definiert
das Erweiterungspaket \verb!pifont! mit der Umgebung \verb!dingautolist! auch ein Äquivalent zur 
Umgebung \verb!enumerate!.

\begin{boxedminipage}{\textwidth}
\texttt{\textbackslash begin\{dingautolist\}\{}\textsl{Zeichen}\texttt{\}}  \enskip \dots\ \enskip \texttt{\textbackslash end\{dingautolist\}} 
\end{boxedminipage}

Im Gegensatz zur \verb!dinglist! wird bei der Umgebung \verb!dingautolist!\index[cmd]{\texttt{dingautolist}}
mit jedem weiteren Aufzählungspunkt 
(\verb!\item!) die Nummer des verwendeten Zeichens um eins erhöht. 
Konkret wird in der Zeichentabelle (siehe Tabelle~\ref{Tabelle_Font_ZapfDingbats}) 
immer ein Zeichen weiter gegangen. So hat jeder Aufzählungspunkt einer Ebene ein anderes
Markierungszeichen.


% \begin{figure}[H]
\begin{minipage}[h]{0.5\textwidth}
\setlength{\parskip}{1em}
\frenchspacing
\begin{Verbatim}[frame=single]
\begin{dingautolist}{172}
  \item Erster Aufzählungspunkt
  \item Zweiter Aufzählungspunkt
  \begin{dingautolist}{182}
    \item Erster Unterpunkt
    \item Zweiter Unterpunkt
  \end{dingautolist}
  \item Dritter Aufzählungspunkt
\end{dingautolist}
\end{Verbatim}
\end{minipage}
\hfill
\begin{minipage}[h]{0.48\textwidth}
\setlength{\parskip}{1em}
\frenchspacing
\begin{dingautolist}{172}
  \item Erster Aufzählungspunkt
  \item Zweiter Aufzählungspunkt
  \begin{dingautolist}{182}
    \item Erster Unterpunkt
    \item Zweiter Unterpunkt
  \end{dingautolist}
  \item Dritter Aufzählungspunkt
\end{dingautolist}
\end{minipage}
% \end{figure}


\section{Kalligrafie}

Das Erweiterungspaket \verb!calligra!~\cite{Calligra_Dokumentation} bietet definiert den Befehl 
\verb!\textcalligra!\index[cmd]{\texttt{\textbackslash textcalligra}}
und bietet dadurch eine einfache Möglichkeit, Texte in einem 
Kalligrafie-Font zu setzen.

\begin{boxedminipage}{\textwidth}
\texttt{\textbackslash textcalligra\{}\textsl{Text}\texttt{\}}
\end{boxedminipage}

\begin{minipage}[h]{0.54\textwidth}
\setlength{\parskip}{1em}
\frenchspacing
\begin{Verbatim}[frame=single]
\textcalligra{Lorem ipsum dolor sit 
amet, consetetur sadipscing elitr, 
sed diam nonumy eirmod tempor 
invidunt ut labore et dolore magna 
aliquyam erat, sed diam voluptua. 
At vero eos et accusam et justo duo 
dolores et ea rebum.}

\textcalligra{0123456789}
\end{Verbatim}
\end{minipage}
\hfill
\begin{minipage}[h]{0.38\textwidth}
\setlength{\parskip}{1em}
\frenchspacing
\textcalligra{Lorem ipsum dolor sit amet, consetetur sadipscing elitr, sed diam nonumy eirmod tempor invidunt ut labore et dolore magna aliquyam erat, sed diam voluptua. At vero eos et accusam et justo duo dolores et ea rebum.}

\textcalligra{0123456789}
\end{minipage}
