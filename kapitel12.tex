\chapter{Präsentationsfolien}
\label{Kapitel_Praesentationsfolien}

Die Verwendung der Dokumentklasse \verb!beamer! ermöglicht auf komfortable Art und Weise die Erstellung optisch ansprechender Präsentationsfolien. 

Ein einfaches Beispiel für ein Grundgerüst einer Quelldatei mit \verb|beamer| zeigt Listing~\ref{erstesbeispielbeamer}. Das Ergebnis dieses Beispiels zeigt Abbildung~\ref{fig_erstesbeispielbeamer}.

\lstinputlisting[caption={Einfaches Grundgerüst einer \LaTeX-Quelldatei mit \texttt{beamer}},label=erstesbeispielbeamer, style=customlatex]{Beispiele/Folien_beamer_einfaches_Beispiel/beamer_einfaches_beispiel.tex}

\begin{figure}[H]
	%     \centering
	% Abschneiden mit trim = liks unten rechts oben
	\fbox{\includegraphics[page=1, clip, width=.98\textwidth]{Beispiele/Folien_beamer_einfaches_Beispiel/beamer_einfaches_beispiel.pdf}}
	\caption{Resultat von Listing~\ref{erstesbeispielbeamer}}
	\label{fig_erstesbeispielbeamer}
\end{figure}

Das grundlegende Layout der Präsentationsfolien definieren sogenannte Themen\footnote{Die (Layout-)Themen von \texttt{beamer} sind immer nach Städten benannt.}. In jedem Foliensatz muss exakt ein Thema definiert sein. 
Die farbliche Gestaltung kann mit Hilfe von Farbthemen\footnote{Die Farbthemen von \texttt{beamer} sind immer nach Tieren benannt.} beeinflusst werden. Eine hilfreiche Übersicht über die verschiedenen Kombinationen aus standardmäßig enthaltenen (Layout-)Themen und Farbthemen bietet~\cite{BeamerThemeMatrixWebseite}. 

Jede Folie, die mit \verb|beamer| erstellt wird, besteht aus einer Umgebung \verb!frame!. Innerhalb der Umgebung kann der Folientitel mit dem Befehl \verb|frametitle| definiert sein. Der Titel wird dem Befehl in geschweiften Klammern als Argument übergeben.

\begin{boxedminipage}{\textwidth}
\verb!\begin{frame}!\newline
\verb!  \frametitle{!\textsl{Folientitel}\verb!}! \newline
\verb!  !\textsl{\dots\ Inhalt der Folie\ \dots} \newline
\verb!\end{frame}!
\end{boxedminipage}

\section{Titelfolie}

Der automatische Satz einer Titelfolie geschieht mit dem Befehl \verb!\titlepage!\index[cmd]{\texttt{\textbackslash titlepage}}.

\begin{boxedminipage}{\textwidth}
\verb!\begin{frame}!\newline
\verb!  \titlepage! \newline
\verb!\end{frame}!
\end{boxedminipage}

Der Befehl \verb!\titlepage! greift zum Satz der Titelfolie auf diejenigen
Argumente zurück, die den Befehlen
\verb|\title|\index[cmd]{\texttt{\textbackslash title}}, 
\verb|\subtitle|\index[cmd]{\texttt{\textbackslash subtitle}}, 
\verb|\author|\index[cmd]{\texttt{\textbackslash author}}, 
\verb|\date|\index[cmd]{\texttt{\textbackslash date}} und
\verb|\institute|\index[cmd]{\texttt{\textbackslash institute}} mit geschweiften Klammern übergeben werden.

Ein Beispiel für das automatische Erzeugen der Titelfolie mit \verb|beamer| zeigt Listing~\ref{Titelfoliebeamer}. Das Ergebnis dieses Beispiels zeigt Abbildung~\ref{fig_Titelfoliebeamer}.

\lstinputlisting[caption={Automatisches Erzeugen der Titelfolie},label=Titelfoliebeamer, style=customlatex]{Beispiele/Folien_beamer_Titelseite/Folien_beamer_beispiel_titel.tex}

\begin{figure}[H]
	%     \centering
	% Abschneiden mit trim = liks unten rechts oben
	\fbox{\includegraphics[page=1, clip, width=.98\textwidth]{Beispiele/Folien_beamer_Titelseite/Folien_beamer_beispiel_titel.pdf}}
	\caption{Resultat von Listing~\ref{Titelfoliebeamer}}
	\label{fig_Titelfoliebeamer}
\end{figure}

Das Beispiel in Listing~\ref{Titelfoliebeamer} zeigt auch eine mögliche Verwendung der Kopf- und Fußzeilen. Im Beispiel wird mit Hilfe des Befehls \verb!\insertnavigation!\index[cmd]{\texttt{\textbackslash insertnavigation}} die Gliederungsstruktur des Foliensatzes in der Kopfzeile ausgegeben. Abschnitte (\verb|section|) sind mit ihren Abschnittsnamen aufgeführt. Jeder Unterabschnitt (\verb!subsection!) und jede Folie sind durch Punkte unter den Abschnittsnamen repräsentiert.  

Im Beispiel ist auch die Ausgabe der 
kleinen Navigationsleiste am unteren Rand 
durch den Befehl \verb!\beamertemplatenavigationsymbolsempty!\index[cmd]{\texttt{\textbackslash beamertemplatenavigationsymbolsempty}}
abgeschaltet.

\section{Automatischer Satz eines Inhaltsverzeichnisses}

Wie bei anderen Dokumentklassen auch, ist es bei \verb|beamer| üblich, Dokumente mit den Befehlen 
\verb|\section|\index[cmd]{\texttt{\textbackslash section}},
\verb|\subsection|\index[cmd]{\texttt{\textbackslash subsection}} und
\verb|\subsubsection|\index[cmd]{\texttt{\textbackslash subsubsection}} zu gliedern. 

Der automatische Satz einer Folie mit der Agenda der Präsentation, also mit dem Inhalt des Foliensatzes, kann mit dem Befehl 
\verb|\tableofcontents|\index[cmd]{\texttt{\textbackslash tableofcontents}} geschehen.

\begin{boxedminipage}{\textwidth}
	\verb!\begin{frame}!\newline
	\verb!  \frametitle{Agenda}! \newline
	\verb!	\tableofcontents[!\textsl{Argument(e)}\verb|]| \newline
	\verb!\end{frame}!
\end{boxedminipage}

Das Verhalten des Befehls kann durch Argumente beeinflusst werden, die in eckigen Klammern übergeben werden.
Sollen beispielsweise nur die Namen der Abschnitte (\verb|sections|) im Inhaltsverzeichnis erscheinen, so realisiert dieses 
der folgende Befehl:

\verb|\tableofcontents[current,hideallsubsections]|

Soll nur die Gliederung des aktuellen Abschnitts (\verb|section|) inklusive seiner Unterabschnitte berücksichtigt werden, 
legt dieses der folgende Befehl fest:

\verb|\tableofcontents[currentsection]|

\section{Inhalte auf Präsentationsfolien setzen}

Klassischerweise werden Inhalte auf Präsentationsfolien mit 
Aufzählungen (siehe Abschnitt~\ref{Abschnitt_Aufzaehlungen}), also mit den Umgebungen \verb!itemize! und \verb!enumerate! gesetzt. 

Sinnvoll kann auch der Gebrauch von farbigen Blöcken sein.
\verb|beamer| stellt standardmäßig Umgebungen für drei verschiedene Blöcke zur Verfügung. Diese sind
\verb|block|\index[cmd]{\texttt{\textbackslash block}}, 
\verb|exampleblock|\index[cmd]{\texttt{\textbackslash exampleblock}} und 
\verb|alertblock|\index[cmd]{\texttt{\textbackslash alertblock}}. Der Titel eines Blocks 
wird der Umgebung als Argument in geschweiften Klammern übergeben.

\begin{boxedminipage}{\textwidth}
	\verb!\begin{block}{!\textsl{Blocktitel}\verb!}! \newline
	\verb!  !\textsl{Blocktext} \newline
	\verb!\end{block}! \\\enskip
	
	\verb!\begin{exampleblock}{!\textsl{Blocktitel}\verb!}! \newline
    \verb!  !\textsl{Blocktext} \newline
    \verb!\end{exampleblock}! \\\enskip

	\verb!\begin{alertblock}{!\textsl{Blocktitel}\verb!}! \newline
    \verb!  !\textsl{Blocktext} \newline
    \verb!\end{alertblock}!
\end{boxedminipage}

Ein Beispiel für die Verwendung der genannten Umgebungen für Aufzählungen und Blöcke zeigt Listing~\ref{Titelbeamerbloecke}. Der Quellcode zeigt nur den relevanten Ausschnitt -- die entsprechende Folie -- aus der Quelldatei. Das Ergebnis dieses Beispiels zeigt Abbildung~\ref{fig_Titelbeamerbloecke}.

\lstinputlisting[caption={Aufzählungen und farbige Blöcke auf Präsentationsfolien},label=Titelbeamerbloecke, style=customlatex, firstline=56, lastline=78,firstnumber=56]{Beispiele/Folien_beamer/Folien_beamer_beispiel.tex}

\begin{figure}[H]
	%     \centering
	% Abschneiden mit trim = liks unten rechts oben
	\fbox{\includegraphics[page=2, clip, width=.98\textwidth]{Beispiele/Folien_beamer/Folien_beamer_beispiel.pdf}}
	\caption{Resultat von Listing~\ref{Titelbeamerbloecke}}
	\label{fig_Titelbeamerbloecke}
\end{figure}

\section{Mehrspaltige Präsentationsfolien setzen}

Ein hilfreiches Werkzeug, um das Layout von Präsentationsfolien zu definiert, ist die Umgebung \verb!columns!\index[cmd]{\texttt{columns}} zum Satz mehrspaltiger Folien.

Innerhalb einer Umgebung \verb|columns| umschließt jede Umgebung \verb|column| den Inhalt einer neuen Spalte.
Die Breite einer Spalte wird als Argument nach dem \verb!\begin{column}! in geschweiften Klammern als absolute oder relative Maßeinheit (siehe Abschnitt~\ref{sec:Massangaben}) oder alternativ mit Hilfe eines Längenbefehls wie zum Beispiel \verb|\textsidth| definiert. Zusätzlich ist es auch möglich, die vertikale Ausrichtung innerhalb jeder Spalte als Option in eckigen Klammern zu definieren. Der Buchstabe \verb|T| steht dabei für \emph{top}, also für eine Ausrichtung am oberen Rand der Spalte, und der Buchstabe \verb|c| steht für \emph{center}, also für eine vertikal zentrierte Ausrichtung des Inhalts. 

Ein Beispiel für eine zweispaltige Folie, bei der beide Spalten gleich groß sind (jeweils die halbe Breite des Textfeldes) und sich in der rechten Spalte eine Abbildung befindet, zeigt Listing~\ref{beamercolumns}. Der Quelltext zeigt nur den relevanten Ausschnitt aus der Quelldatei. Das Ergebnis dieses Beispiels zeigt Abbildung~\ref{fig_beamercolumns}. 

\lstinputlisting[caption={Mehrspaltige Präsentationsfolien ermöglicht die Umgebung \texttt{columns}},label=beamercolumns, style=customlatex, firstline=83, lastline=106,firstnumber=83]{Beispiele/Folien_beamer/Folien_beamer_beispiel.tex}

\begin{figure}[H]
	%     \centering
	% Abschneiden mit trim = liks unten rechts oben
	\fbox{\includegraphics[page=3, clip, width=.98\textwidth]{Beispiele/Folien_beamer/Folien_beamer_beispiel.pdf}}
	\caption{Resultat von Listing~\ref{beamercolumns}}
	\label{fig_beamercolumns}
\end{figure}
