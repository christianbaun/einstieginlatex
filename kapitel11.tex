\chapter{Literaturverzeichnis}
\label{Kapitel_Literaturverzeichnis}
\index{Literaturverzeichnis}

Das Einbinden eines Literaturverzeichnisses in Dokumente kann auf verschiedene Arten geschehen. Unterschiede in den Vorgehensweisen bestehen in erster Linie hinsichtlich des Automatisierungsgrads. In diesem Dokument ist das Erstellen von Literaturverzeichnissen mit der Umgebung \verb!thebibliography! und mit Bib\TeX\ beschrieben.

Ganz unabhängig von der Art und Weise, wie das Literaturverzeichnis realisiert wird, geschieht der Verweis auf Einträge im Literaturverzeichnis mit dem Befehl \verb|\cite|.\index[cmd]{\texttt{\textbackslash cite}}

\begin{boxedminipage}{\textwidth}
	\texttt{\textbackslash cite[\textsl{Text}]\{\textsl{Schlüssel}\}} 
\end{boxedminipage}

Der zwingend nötige Parameter \textsl{Schlüssel} verweist auf einen oder mehrere Einträge im Literaturverzeichnis. Der Inhalt des optionalen Parameters  \textsl{Text} wird zusätzlich ausgegeben und kann eine weiterführende Information für den Leser enthalten wie zum Beispiel \glqq S.\,42-46\grqq\ oder \glqq Kapitel 5\grqq.

Soll auf mehrere Einträge im Literaturverzeichnis gleichzeitig referenziert werden, müssen die Angaben im Parameter \textsl{Schlüssel} durch Kommata voneinander getrennt sein.

\section{Manueller Satz des Literaturverzeichnisses}
\label{Abschnitt_thebibliography}

Das Einfügen des des Literaturverzeichnisses in das Dokument geschieht mit einer Umgebung \verb!thebibliography! am Ende des \LaTeX-Quelltextes.
\index[cmd]{\texttt{thebibliography}}

\begin{boxedminipage}{\textwidth}
	\texttt{\textbackslash begin\{thebibliography\}\{\textsl{Markierungsbreite}\}} \enskip \dots\ \enskip \texttt{\textbackslash end\{thebibliography\}} 
\end{boxedminipage}

Die Breite (Anzahl der Zeichen) der einzelnen Labels (Markierungen) im Literaturverzeichniss definiert der Inhalt des Parameters \textsl{Markierungsbreite}. Dieser kann eine beliebige Zeichenkette enthalten (z.B. \verb|999|)~\cite{voss2007referenz}.

Die einzelnen Einträge im Literaturverzeichnis realisiert der Befehl \verb|\bibitem| innerhalb der Umgebung \verb!thebibliography!. 
\index[cmd]{\texttt{\textbackslash bibitem}}

\begin{boxedminipage}{\textwidth}
	\texttt{\textbackslash bibitem[\textsl{Markierung}]\{\textsl{Schlüssel}\} \textsl{Text}} 
\end{boxedminipage}

Ein Beispiel für eine \verb!thebibliography!-Umgebung mit drei Einträgen enthält Listing~\ref{listing_thebibliography_beispiel}. Das resultierende Literaturverzeichnis zeigt Abbildung~\ref{fig_thebibliography_beispiel}.

\begin{lstlisting}[caption={Ein Literaturverzeichnis mit der Umgebung \texttt{thebibliography}},label=listing_thebibliography_beispiel, style=customlatex]
\begin{thebibliography}{9}
\bibitem[Go93]{GoMiSa93} 
Michel Goossens, Frank Mittelbach and Alexander Samarin. 
\emph{The \LaTeX\ Companion}. 
Addison-Wesley, Reading, Massachusetts, 1993.
\bibitem[La94]{Lamport94}
Leslie Lamport,	\emph{\LaTeX: A Document Preparation System},	
Addison Wesley, Massachusetts, 2nd edition, 1994.
\bibitem[Kn86]{Knuth96} 
Donald E. Knuth, \emph{Computers and Typesetting -- Volume A: 
The \TeX book}, Addison Wesley, 1986.
\end{thebibliography}
\end{lstlisting}

\begin{figure}[H]
	%     \centering
	% Abschneiden mit trim = liks unten rechts oben
	\fbox{\includegraphics[page=1, clip, trim=2cm 22cm 2cm 2cm, width=.98\textwidth]{Beispiele/thebibliography/thebibliography_beispiel.pdf}}
	\caption{Resultat von Listing~\ref{listing_thebibliography_beispiel}}
	\label{fig_thebibliography_beispiel}
\end{figure}

Ein Unterschied zum Satz des Literaturverzeichnisses mit Bib\TeX\ ist, das es für das Aussehen und den Umfang des Literaturverzeichnisses keine Rolle spielt, ob im Dokument auf einzelne Einträge referenziert wird. 
Das Literaturverzeichnis im fertig gesetzten Dokument enthält immer alle im Quelltext aufgeführten Einträge.

\section{Autmoatisierter Satz des Literaturverzeichnisses}
\label{Abschnitt_bibtex}
\index{Bib\TeX}

Vielfältige Möglichkeiten ein Literaturverzeichniss einzufügen und dessen Formatierung zu beeinflussen bietet \LaTeX\ in Zusammenarbeit mit dem Programm Bib\TeX\ und den Befehlen \verb|\bibliography| und \verb|\bibliographystyle|. Die Kombination aus diesen Programmen und Befehlen ermöglicht es, Quellen aus einer oder mehreren Literaturdatenbanken herauszusuchen und beispielsweise nach dem Nachnamen des ersten Autors oder gemäß der Reihenfolge der Referenzierung im Dokument zu sortieren. 

Die Einbindung einer Literaturdatenbank (\verb|.bib|-Datei) geschieht mit dem Befehl \verb|\bibliography|\index[cmd]{\texttt{\textbackslash bibliography}}. Der Dateiname wird ohne Dateiendung angegeben.

\begin{boxedminipage}{\textwidth}
	\texttt{\textbackslash bibliography\{\textsl{Dateiname}\}} 
\end{boxedminipage}

Sollen mehrere Literaturdatenbanken eingebunden werden, müssen die entsprechenden \verb|.bib|-Dateien durch Kommata voneinander abgetrennt sein. Auch hierbei werden die Dateinamen ohne Dateiendung angegeben.

\begin{boxedminipage}{\textwidth}
	\texttt{\textbackslash bibliography\{\textsl{Dateiname1},\textsl{Dateiname2},\textsl{Dateiname3},...\}} 
\end{boxedminipage}

Die Definition des Stils des Literaturverzeichnisses geschieht mit dem Befehl \verb|\bibliographystyle|\index[cmd]{\texttt{\textbackslash bibliographystyle}}.

\begin{boxedminipage}{\textwidth}
	\texttt{\textbackslash bibliographystyle\{\textsl{Stil}\}} 
\end{boxedminipage}

Der Befehl darf an (fast) beliebiger Stelle in der \verb!.tex!-Quelldatei stehen.

Es existiert eine sehr große Anzahl an Stilen. Umfangreiche Übersichten enthalten unter anderem~\cite{voss2007referenz} und~\cite{BibtexStylesShareLaTeXWebseite}. Eine Auswahl an häufig verwendeten Stilen enthält Tabelle~\ref{Tabelle_Stile_Literaturverzeichnisse}. 

\begin{table}[h!tb]
	\centering
	\caption[Auswahl an Stilen für Literaturverzeichnisse]{Auswahl an Stilen für Literaturverzeichnisse~\cite{Wikibooks_LaTeX_Woerterbuch}}
	\label{Tabelle_Stile_Literaturverzeichnisse}       % Give a unique label
	\begin{tabularx}{\textwidth}{lp{3cm}p{7.5cm}}
		\hline
		Stilname & Sortierung & Referenzierung  \\
		\hline
		\texttt{abbrv} & Erstautor & [IdNr] (Vornamen nur als Initialen) \\
		\texttt{alpha} & Erstautor & bis zu drei Anfangsbuchstaben der Autorennamen, evtl. ein hochgestelltes Pluszeichen, bei mehr als drei Autoren, Jahresangabe zweistellig \\			
		\texttt{apalike} & Erstautor & [Autorenname(n), Jahr] \\
		\texttt{plain}   & Erstautor & [IdNr] (Vornamen werden ausgeschrieben) \\
        \texttt{plainyr} & Referenzart, Jahr   & [IdNr] (Vornamen werden ausgeschrieben) \\
        \texttt{unsrt}   & Reihenfolge der Referenzierung & [IdNr] \\
		\hline
	\end{tabularx}
\end{table}

Für jeden Stil muss eine Datei gleichen Namens mit der Dateiendung \verb!.bst! existieren. Dort ist das Layout des Stils definiert.
Bei der \LaTeX-Distribution \TeX~Live befinden sich die \verb!.bst!-Dateien im Unterverzeichnis \verb|texlive/texmf-dist/bibtex/bst/|.


Bei der Arbeit mit Bib\TeX\ müssen die einzelnen Literatureinträge in einer oder mehreren Dateien mit der Dateiendung
\verb|.bib| vorliegen. Je nach Art des jeweiligen Eintrags (zum Beispiel Bücher, Artikel, technische Dokumentationen, etc.) sind bestimmte Angaben (\glqq Felder\grqq) zum Eintrag zwingend nötig oder optional möglich. Häufig verwendete Typen von Einträgen in Literaturdatenbanken sind \texttt{article}, \texttt{book}, \texttt{booklet}, \texttt{conference}, \texttt{inbook}, \texttt{incollection}, \texttt{inproceedings}, \texttt{manual}, \texttt{mastersthesis}, \texttt{misc}, \texttt{phdthesis}, \texttt{proceedings}, \texttt{techreport} und \texttt{unpublished}.

Jedem Eintrag ist ein Klammeraffe\index{Klammeraffe} (\verb!@!) vorangestellt. Es ist egal, ob das Schlüsselwort eines
Eintrags mit Klein- oder Großbuchstaben 
geschrieben ist. Jede der folgenden Schreibweisen ist somit korrekt: \verb!@book!, \verb!@BOOK!, 
\verb!@bOOk!, \verb!@BoOk!, etc.

Auf den Typ eines Eintrags folgt immer ein Paar 
geschweifter Klammern, und dazwischen befinden sich ein Schlüsselwort\index{Schlüsselwort} und 
die einzelnen Felder.
Das Schlüsselwort kann aus einer fast beliebigen
Folge von Buchstaben, 
Zahlen und Zeichen jeder Art bestehen. 
Kommata sind im Schlüsselwort tabu. 
Idealerweise ist das Schlüsselwort nicht nur eindeutig, sondern beim Lesen des Quelltextes auch nachvollziehbar.

Die einzelnen Felder bestehen 
aus jeweils einem Feldnamen
und einer Information (Feldtext).
Mögliche Felder sind u.a. \texttt{address}, \texttt{author}, \texttt{booktitle}, \texttt{chapter}, \texttt{edition}, \texttt{editor}, \texttt{isbn}, \texttt{journal}, \texttt{volume}, \texttt{month}, \texttt{note}, \texttt{number}, \texttt{organization}, \texttt{pages}, \texttt{series}, \texttt{school}, \texttt{title} und \texttt{year}. Es werden allerdings nicht alle Felder bei allen Arten von Einträgen berücksichtigt und bei allen Arten von Einträgen sind bestimmten Felder zwingend erforderlich. Eine Übrsicht über diese Typen von Einträgen in Literaturdatenbanken und deren zwingende sowie optionale Felder enthält Abschnitt~\ref{Abschnitt_Eintraege_Bibtex_Felder}.

Zwischen Feldname und Feldtext befindet sich ein \verb!=!. Der Feldtext ist von geschweiften Klammern oder alternativ mit Anführungsstrichen (\verb!" ... "!) umschlossen. Besteht der Feldtext nur aus Ziffern, sind die geschweiften Klammern bzw. Anführungsstriche unnötig.
Die einzelnen Felder sind durch Kommata voneinander getrennt.

Syntaktisch korrekt sehen Einträge in Literaturdatenbanken folgendermaßen aus: 

\begin{boxedminipage}{\textwidth}
\verb!@!\textsl{Typ}\verb!{!\textsl{Schlüsselwort}\verb!,!\newline
\verb!    !\textsl{Feldname}\verb! = {!\textsl{Feldtext}\verb!},! \newline
\verb!    !\textsl{Feldname}\verb! = {!\textsl{Feldtext}\verb!},! \newline
\verb!    !\textsl{Feldname}\verb! = {!\textsl{Feldtext}\verb!},! \newline
\verb!    !\verb!...! \newline
\verb!    !\textsl{Feldname}\verb! = {!\textsl{Feldtext}\verb!}! \newline
\verb!}!
\end{boxedminipage}

Zwei Beispiele für Bib\TeX-Einträge enthält Listing~\ref{Bibtext_beispiele_listing}.

\begin{lstlisting}[caption={Einige Beispeile für Bib\TeX-Einträge},label=Bibtext_beispiele_listing, style=customlatex]
@article{InformatikSpektrum2018,
author    = {Christian Baun and
             Henry-Norbert Cocos and
             Rosa-Maria Spanou},
title     = {Erfahrungen beim Aufbau von großen Clustern aus
             Einplatinencomputern für Forschung und Lehre},
journal   = {Informatik Spektrum},
volume    = {41},
number    = {3},
pages     = {189--199},
year      = {2018},
doi       = {10.1007/s00287-017-1083-9}
}

@book{ComputernetzeKompaktBuch2018,
author    = {Christian Baun},
title     = {Computernetze kompakt},
publisher = {Springer Vieweg},
edition   = {4},
year      = {2018},
isbn      = {978-3-662-57468-3}
}
\end{lstlisting}


In den meisten Fällen ist die effizienteste Vorgehensweise fertige Bib\TeX-Einträge zu den gewünschten Quellen von Online-Angeboten wie Google Scholar~\cite{GoogleScholarWebseite} oder dblp~\cite{DBLPWebseite} zu beziehen, anstatt diese selbst zu erstellen.

Weitere Möglichkeiten, um das Aussehen des Literaturverzeichnisses und der Referenzen zu definieren, bietet das Erweiterungpaket \verb|natbib|~\cite{natbibDoku}\index[cmd]{\texttt{natbib}}. Diese bietet auch weitere Felder wie zum Beispiel \texttt{doi}, \texttt{issn} und \texttt{url}.

\section{Funktionsweise von Bib\TeX}
\label{Abschnitt_Funktionsweise_Bibtex}

Beim Durchlauf des \LaTeX-Compilers erzeugt dieser eine Textdatei mit der Endung \verb|.aux|. Ein nachfolgender Durchlauf von Bib\TeX\ erzeugt aus der \verb|.aux|-Datei mit Hilfe der \verb|.bib|-Datei(en) und der \verb|.bst|-Datei eine weitere Textdatei mit der Endung \verb|.bbl|. Diese letzte Datei enthält das mit \verb|thebibliography|\index[cmd]{\texttt{thebibliography}} realisierte Literaturverzeichnis mit den im Dokument angeforderten Einträgen. Beim nächsten Durchlauf des \LaTeX-Compilers fügt dieser das Literaturverzeichnis in das fertige Dokument ein.

\section{Einträgen in Literaturdatenbanken und deren Felder}
\label{Abschnitt_Eintraege_Bibtex_Felder}

Dieser Abschnitt enthält eine Übersicht über einige gängige Typen von Einträgen in Literaturdatenbanken und deren zwingende sowie optionale Felder~\cite{bibtexWikipediaWebseite,Wikibooks_LaTeX_Kompendium}.

\verb!@article! Literaturangabe für einen Artikel aus einer Zeitschrift, oder Zeitung.

\begin{tabular}{p{2.8cm}p{10cm}}
	 \textsl{zwingende Felder} & \texttt{author}, \texttt{title}, \texttt{journal}, \texttt{year}\\
	 \textsl{optionale Felder} & \texttt{volume}, \texttt{number}, \texttt{month}, \texttt{note}, \texttt{pages}\\
\end{tabular}

\verb!@book! Literaturangabe für ein Buch aus einem Verlag.

\begin{tabular}{p{2.8cm}p{10cm}}
	 \textsl{zwingende Felder} & \texttt{author} oder \texttt{editor}, \texttt{title}, \texttt{publisher}, \texttt{year}\\
	 \textsl{optionale Felder} & \texttt{volume} oder \texttt{number}, \texttt{series}, \texttt{address}, \texttt{edition}, \texttt{month}, \texttt{note}, \texttt{isbn} \\
\end{tabular}

\verb!@booklet! Literaturangabe für ein Buch (gebundenes Druckwerk) ohne Verlag.

\begin{tabular}{p{2.8cm}p{10cm}}
	 \textsl{zwingende Felder} & \texttt{title}\\
	 \textsl{optionale Felder} & \texttt{author}, \texttt{howpublished}, \texttt{address}, \texttt{month}, \texttt{year}, \texttt{note}\\
\end{tabular}

\verb!@conference! identisch mit \verb!@inproceedings!.

\verb!@inbook! Literaturangabe für einen Buchauszug, z.~B. 
ein Kapitel oder ein paar Seiten.

\begin{tabular}{p{2.8cm}p{10cm}}
	 \textsl{zwingende Felder} & \texttt{author} oder \texttt{editor}, \texttt{title}, \texttt{chapter} und/oder \texttt{pages}, \texttt{publisher}, \texttt{year}\\
	 \textsl{optionale Felder} & \texttt{volume} oder \texttt{number}, \texttt{series}, \texttt{type}, \texttt{address}, \texttt{edition}, \texttt{month}, \texttt{note}\\
\end{tabular}

\verb!@incollection! Literaturangabe für einen 
Buchauszug mit einem eigenen Titel, z.~B. einen Aufsatz in einem Sammelband.

\begin{tabular}{p{2.8cm}p{10cm}}
	 \textsl{zwingende Felder} & \texttt{author}, \texttt{title}, \texttt{booktitle}, \texttt{publisher}, \texttt{year}\\
	 \textsl{optionale Felder} & \texttt{editor}, \texttt{volume} oder \texttt{number}, \texttt{type}, \texttt{series}, \texttt{edition}, \texttt{chapter}, \texttt{pages}, \texttt{address}, \texttt{month}, \texttt{note}\\
\end{tabular}

\verb!@inproceedings! Literaturangabe für einen 
Artikel aus einem Konferenzbericht.

\begin{tabular}{p{2.8cm}p{10cm}}
	 \textsl{zwingende Felder} & \texttt{author}, \texttt{title}, \texttt{booktitle}, \texttt{year}\\
	 \textsl{optionale Felder} & \texttt{editor}, \texttt{volume} oder \texttt{number}, \texttt{organisation}, \texttt{series}, \texttt{pages}, \texttt{publisher}, \texttt{address}, \texttt{month}, \texttt{note}\\
\end{tabular}

\verb!@manual! Literaturangabe für eine technische Dokumentation.

\begin{tabular}{p{2.8cm}p{10cm}}
	 \textsl{zwingende Felder} & \texttt{title}, \texttt{address}, \texttt{year} \\
	 \textsl{optionale Felder} & \texttt{author}, \texttt{organisation}, \texttt{edition}, \texttt{month}, \texttt{note}\\
\end{tabular}

\verb!@mastersthesis! Literaturangabe für eine Diplomarbeit oder Masterthesis.

\begin{tabular}{p{2.8cm}p{10cm}}
	 \textsl{zwingende Felder} & \texttt{author}, \texttt{title}, \texttt{school}, \texttt{year}\\
	 \textsl{optionale Felder} & \texttt{address}, \texttt{month}, \texttt{note}, \texttt{type}\\
\end{tabular}

\verb!@misc! Literaturangabe die zu keiner dieser Eingabetypen passt (z.~B. eine Web-Seite).

\begin{tabular}{p{2.8cm}p{10cm}}
	 \textsl{zwingende Felder} & mindestes eines der \textsl{optionalen Felder}\\
	 \textsl{optionale Felder} & \texttt{author}, \texttt{title}, \texttt{howpublished}, \texttt{month}, \texttt{year}, \texttt{note}\\
\end{tabular}

\verb!@phdthesis! Literaturangabe für eine Doktorarbeit (Promotion).

\begin{tabular}{p{3cm}p{9cm}}
	 \textsl{zwingende Felder} & \texttt{author}, \texttt{title}, \texttt{school}, \texttt{year}\\
	 \textsl{optionale Felder} & \texttt{address}, \texttt{month}, \texttt{note}, \texttt{type}\\
\end{tabular}

\verb!@proceedings! Literaturangabe für einen Konferenzbericht.

\begin{tabular}{p{2.8cm}p{10cm}}
	 \textsl{zwingende Felder} & \texttt{title}, \texttt{year}\\
	 \textsl{optionale Felder} & \texttt{editor}, \texttt{publisher}, \texttt{volume} oder \texttt{number}, \texttt{organisation}, \texttt{series}, \texttt{address}, \texttt{month}, \texttt{note} \\
\end{tabular}

\verb!@techreport! Literaturangabe für einen Bericht einer Hochschule oder eines Instituts.

\begin{tabular}{p{2.8cm}p{10cm}}
	 \textsl{zwingende Felder} & \texttt{author}, \texttt{title}, \texttt{institution}, \texttt{year}\\
	 \textsl{optionale Felder} & \texttt{type}, \texttt{number}, \texttt{address}, \texttt{month}, \texttt{note} \\
\end{tabular}

\verb!@unpublished! Literaturangabe für eine unveröffentlichte Arbeit mit einem Autor und einem Titel.

\begin{tabular}{p{2.8cm}p{10cm}}
	 \textsl{zwingende Felder} & \texttt{author}, \texttt{title}, \texttt{note}\\
	 \textsl{optionale Felder} & \texttt{month}, \texttt{year} \\
\end{tabular}






\section{Das Aussehen des Literaturverzeichnisses anpassen}
\label{Abschnitt_Literaturverzeichnis_anpassen}

Die Änderung der Überschrift eines Literaturverzeichnisses geschieht mit folgendem Befehl:

\begin{boxedminipage}{\textwidth}
\texttt{\textbackslash renewcommand\{\textbackslash refname\}\{Literaturverzeichnis\}}
\end{boxedminipage}

Einige Autoren bevorzugen anstatt \glqq Literaturverzeichnis\grqq\ beispielsweise die Überschrift \glqq Quelle\grqq\ oder einfach \grqq Literatur\grqq.

Das Hinzufügen eines Eintrags in das Inhaltsverzeichnis realisiert der folgende Befehl:


\begin{boxedminipage}{\textwidth}
\texttt{\textbackslash addcontentsline\{toc\}\{chapter\}\{Literaturverzeichnis\}}
\end{boxedminipage}




\section{Die Literaturdatenbank mit der \LaTeX-Quelldatei erzeugen}
\label{Abschnitt_Literaturdatenbank_Quelldatei_erzeugen}

Mit dem Erweiterungspaket \verb|filecontents|\index[cmd]{\texttt{filecontents}} ist es möglich aus einer \LaTeX-Quelldatei heraus eine andere Datei zu erstellen und mit Inhalt zu befüllen bzw. deren Inhalt zu überschreiben. Sobald das Paket mit dem Befehl \verb!\usepackage{textcomp}! in der Präambel der \LaTeX-Quelldatei eingebunden wurde, ist die gleichnamige Umgebung \verb|filecontents| verfügbar.

Damit ist es z.~B. möglich eine Bib\TeX-Literaturdatenbank aus der Quelldatei heraus zu erzeugen.
Listing~\ref{bibtex_eingebettet_beispiel} zeigt exemplarisch die Vorgehensweise. Das Ergebnis des
Beispiels zeigt Abbildung~\ref{fig_bibtex_eingebettet_beispiel}.

Beim Start der Umgebung \verb|filecontents| wird in geschweiften Klammern als Parameter der Dateiname der zu erzeugenden bzw. zu überschreibenden Datei definiert. Der in Listing~\ref{bibtex_eingebettet_beispiel} verwendete Befehl \verb|\jobname|\index[cmd]{\texttt{\textbackslash jobname}} fügt an der Stelle seines Aufrufs den Namen der \LaTeX-Quelldatei ohne die Dateiendung ein. Auf diese Art und Weise wird eine \verb|.bib|-Datei mit dem gleichen Dateinamen wie die \LaTeX-Quelldatei erzeugt.

\lstinputlisting[caption={Eine Bib\TeX-Literaturdatenbank aus der \LaTeX-Quelldatei heraus erzeugen},label=bibtex_eingebettet_beispiel, style=customlatex]{Beispiele/Bibtex_eingebettet/bibtex_eingebettet_beispiel.tex}

\begin{figure}[H]
	%     \centering
	% Abschneiden mit trim = liks unten rechts oben
	\fbox{\includegraphics[page=1, clip, trim=2cm 22.5cm 2cm 2cm, width=.98\textwidth]{Beispiele/Bibtex_eingebettet/bibtex_eingebettet_beispiel.pdf}}
	\caption{Resultat von Listing~\ref{bibtex_eingebettet_beispiel}}
	\label{fig_bibtex_eingebettet_beispiel}
\end{figure}
