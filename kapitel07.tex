\chapter{Bilder einfügen}
\label{KapitelBilder}
\index{Abbildungen}
\index{Bilder}
\index{Grafiken}

Um Bilder in den Dateiformaten \verb!.jpg!, \verb!.png! oder \verb!.pdf! in Dokumente einbinden zu können, muss mit das Erweiterungspaket \verb!graphicx! mit dem Befehl \verb!\usepackage{graphicx}! in der Präambel der \verb!.tex!-Quelldatei eingebunden sein.

Das beste optische Ergebnis liefern Abbildungen, die als Vektorgrafiken\index{Vektorgrafik} vorliegen. Besonders bei selbst erstellten Diagrammen ist das problemlos möglich. Im Gegensatz zu Rastergrafiken\index{Rastergrafik} ist das Skalieren von Vektorgrafiken ohne Qualitätsverlust stufenlos möglich. Dateiformate für Vektorgrafiken sind u.a. \verb!eps!, \verb!ps!, \verb!pdf! und \verb!svg!. Dateiformate für Rastergrafiken sind u.a. \verb!bmp!, \verb!gif!, \verb!jpg! und \verb!png!.

Das Einbindern der Bilder geschieht mit dem Befehl \verb!\includegraphics!\index[cmd]{\texttt{\textbackslash includegraphics}}.

\fbox{\texttt{\textbackslash includegraphics[}\textsl{Optionen}\texttt{]\{}\textsl{Dateiname}\texttt{\}}}

Optionen brauchen nicht zwingend angegeben werden. Allerdings kann es 
passieren, das das Bild dann zu groß oder zu
klein im fertigen Dokument erscheint. In den meisten Fällen ist es darum sinnvoll, zumindest mit
\verb!width=!\textsl{Breite} die Bildbreite oder mit 
\verb!height=!\textsl{Höhe} die Bildhöhe festzulegen.

Das Beispiel in Abbildung~\ref{Beispiel_includegraphics1} zeigt das Linux-Maskottchen Tux von Larry Ewing (Bildlizenz: CC0), skaliert auf eine Bildhöhe von 4\,cm.

\begin{figure}[H]
\begin{minipage}[c]{0.34\textwidth}
\setlength{\parskip}{1em}
\centering
\includegraphics[height=4cm]{Tux.pdf}
\end{minipage}
\hfill
\begin{minipage}[c]{0.64\textwidth}
\setlength{\parskip}{1em}
\begin{lstlisting}[label=includegraphicsbeispiel, style=customlatex]
\includegraphics[height=4cm]{Tux.pdf}
\end{lstlisting}
\end{minipage}
\caption{Abbildungen skalieren die Optionen \texttt{height} oder alternativ \texttt{width}}
\label{Beispiel_includegraphics1}
\end{figure}

Soll eine Abbildung genauso breit sein wie das Textfeld, können Autoren das mit \verb!width=\textwidth! realisieren. Das folgende Beispiel würde die Abbildung \verb!bilddatei.pdf! so skalieren, das die Bildbreite der halben Breite des Textes entspricht.

\verb!\includegraphics[width=.5\textwidth]{bilddatei.pdf}!

Alternativ kann auch die Bildbreite fest inklusive einer Maßeinheit 
(siehe Abschnitt~\ref{sec:Massangaben}) definiert sein.

Das Drehen von Bildern ermöglicht der Parameter \verb!angle=!\textsl{Winkel}. 

Im Beispiel in Abbildung~\ref{Beispiel_includegraphics2} ist das Linux-Maskottchen auf die Bildhöhe 3\,cm skaliert und um 270 Grad gedreht.

\begin{figure}[H]
\begin{minipage}[c]{0.34\textwidth}
\setlength{\parskip}{1em}
\centering
\includegraphics[height=3cm,angle=270]{Tux.pdf}
\end{minipage}
\hfill
\begin{minipage}[c]{0.64\textwidth}
\setlength{\parskip}{1em}
\begin{lstlisting}[label=includegraphicsanglebeispiel, style=customlatex]
\includegraphics[height=3cm,angle=270]{Tux.pdf}
\end{lstlisting}
\end{minipage}
\caption{Abbildungen drehen ermöglicht die Option \texttt{angle}}
\label{Beispiel_includegraphics2}
\end{figure}

\section{Bilder als Gleitobjekte}
\label{sec:bilder_gleitobjekte}
\index{Abbildungen!Gleitobjekt}
\index{Bilder!Gleitobjekt}
\index{Grafiken!Gleitobjekt}
\index{Gleitobjekt}
\index[cmd]{\texttt{\textbackslash figure}}

Meist ist es sinnvoll, Abbildungen mit der Umgebung \verb!figure! zu umschließen. Dadurch werden sie zu sogenannten Gleitobjekten. Vorteile von Gleitobjekten sind die automatisierte Positionierung sowie die Möglichkeit Bildunterschriften und Bezeichner zuweisen zu können. Ein mögliches Beispiel ist:

\begin{Verbatim}[frame=single]
\begin{figure}[htb]
\includegraphics[width=7cm]{Dateiname}
\caption{Hier ist die Bildunterschrift definiert}
\label{fig:eindeutiger_bezeichner}
\end{figure}
\end{Verbatim}

Außer \verb!figure!-Gleitobjekten gibt es unter anderem auch \verb!table!-Gleitobjekte (siehe Abschnitt~\ref{sec:tabellen_gleitobjekte}). deren Platzierungsparameter innerhalb der eckigen Klammern sind identisch und auf Seite~\pageref{sec:tabellen_gleitobjekte} ausführlich beschrieben.


\section{Auf Bilder referenzieren und Bildunterschriften definieren}
\label{sec:bilder_referenzieren}

Befindet sich eine Abbildung innerhalb eines \verb!figure!-Gleitobjekts, kann der Abbildung mit dem Befehl \verb!\label{!\textsl{Name}\verb!}! ein Bezeichner zugewiesen werden. In diesem Fall kann an jeder beliebigen Stelle im Dokument mit dem Befehl \verb!\ref{!\textsl{Name}\verb!}! auf die Abbildung verwiesen werden.

Mit dem Befehl \verb!\caption{!\textsl{Bildunterschrift}\verb!}!\index[cmd]{\texttt{\textbackslash caption}} geschieht die Definition der Bildunterschrift\index{Bildunterschrift}. Auch diese Fähigkeit setzt die Einbettung der Abbildung in einem \verb!figure!-Gleitobjekt voraus. 


\section{Pfade der Bilddateien definieren}

Beim Einbinden einer Abbildung mit dem Befehl \verb!\includegraphics! kann der Pfad zur Bilddatei inklusive des Pfades relativ von der \verb!.tex!-Quelldatei oder absolut von der Wurzel des Dateisystems angegeben sein. Einige Beispiele enthält Tabelle~\ref{Tabelle_Relative_und_Absolute_Pfade}.

\begin{table}[h!tb]
\centering
\caption{Beispiele für relative und absolute Pfade}
\label{Tabelle_Relative_und_Absolute_Pfade}
\begin{tabular}{ccl}
\hline
\textbf{Pfad} & \textbf{Betriebssystem} & \textbf{Beispiel} \\
\hline
relativ & Linux/UNIX   & \texttt{\textbackslash includegraphics[}\textsl{...}\texttt{]\{}\textsl{pfad/zur/bilddatei}\texttt{\}} \\
absolut & Linux/UNIX   & \texttt{\textbackslash includegraphics[}\textsl{...}\texttt{]\{}\textsl{/pfad/zur/bilddatei}\texttt{\}} \\
relativ & Windows & \texttt{\textbackslash includegraphics[}\textsl{...}\texttt{]\{}\textsl{pfad\textbackslash zur\textbackslash bilddatei}\texttt{\}}  \\
absolut & Windows & \texttt{\textbackslash includegraphics[}\textsl{...}\texttt{]\{}\textsl{C:\textbackslash pfad\textbackslash zur\textbackslash bilddatei}\texttt{\}}  \\
\hline
\end{tabular}
\end{table}

Dieses Vorgehen hat den Nachteil, das wenn sich der Pfad zu den eingefügten Bilddateien  einmal ändert, sind an mehreren Stellen im Quelltext Änderungen nötig. Aus diesem Grund ist es sinnvoll, den oder die Pfade zu den verwendeten Bilddateien zentral an einer Stelle in der Präambel der \verb!.tex!-Quelldatei zu definieren. Dieses geschieht mit dem Befehl \verb!\graphicspath!\index[cmd]{\texttt{\textbackslash graphicspath}}. 

\fbox{\texttt{\textbackslash graphicspath\{\{}\textsl{Verzeichnis1/}\texttt{\}\{}\textsl{Verzeichnis2/}\texttt{\}...\{}\textsl{Verzeichnisn/}\texttt{\}\}}}

Jeder als Argument übergebene Pfad (egal ob es sich dabei um einen relativen oder um einen absoluten Pfad handelt), sollte unter Linux mit einem Slash enden und unter Windows mit einem Backslash. Zudem muss jeder Pfad von geschweiften Klammern umschlossen sein.
