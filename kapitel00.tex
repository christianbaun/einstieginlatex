\chapter{Vorwort}

Wer sich schon einige Zeit mit dem Textsatz und \LaTeX\ beschäftigt, und zum ersten Mal auf dieses Werk stößt, der wird sich vielleicht folgende berechtige Frage stellen:

\begin{quote}
\glqq Warum hat sich der Autor die Mühe gemacht dieses Werk zu schreiben?\grqq
\end{quote}

In der Tat sind in den letzten Jahrzehnten zahlreiche exzellente Bücher in deutscher~\cite{Kopka2000,GoossensMittelachSamarin2000,voss2016einfuhrung} und englischer~\cite{mittelbach2004latex,kopka2003guide} Sprache zu diesem Thema erschienen und wer in den gängigen Suchmaschinen nach einer guten und freien Einführung zum Thema sucht, der wird an vielen Stellen~\cite{KrauseLink,JuergensFeuerstackLink,RichterTorstenLink,MarxBueckerLink,NagelLink,GitterLink} fündig.

Es ist auch nicht das Ziel, das dieses Werk einmal eine möglichst umfangreiche Dokumentation zum Textsatz mit \LaTeX\ wird, das andere Werke bzgl. des Umfangs in den Schatten stellt. 

Die Motivation hinter der Entstehung dieses Werks ergab sich in erster Linie daraus, das es nur wenige einführende Bücher zu \LaTeX\ gibt, deren Quelltext frei ist und deren Lizenz so viele Freiheiten bietet, wie die für dieses Werk verwendete Lizenz.

Den Quellcode des vorliegenden Dokuments finden Sie unter dieser Webadresse:\\
\url{https://github.com/christianbaun/einstieginlatex}

Das Werk ist lizenziert unter der Lizenz Creative Commons mit den Einschränkungen \glqq Namensnennung\grqq\ und \glqq Weitergabe unter gleichen Bedingungen\grqq\ in der Version 3.0 für Deutschland~\cite{CC-BY-SA-3.0License}.

Wenn Sie Fehler oder Ungenauigkeiten in diesem Dokument finden oder Verbesserungsvorschläge haben, dann zögern Sie bitte nicht mir zu schreiben. 

\begin{flushright} 
Frankfurt am Main\hfill Prof.~Dr.~Christian Baun\\
September 2018
\end{flushright}
