\chapter{Tabellen}
\label{Kapitel_Tabellen}
\index{Tabellen}

\LaTeX\ bietet zahlreiche Umgebungen zum Satz von Tabellen. Aus Platzgründen kann dieses Dokument aber nur eine Auswahl, nämlich die Umgebungen \verb!tabular!
und \verb!tabularx! berücksichtigen.



\section{Die Umgebung \texttt{tabular}}
\index[cmd]{\texttt{tabular}}

Die Umgebung \verb!tabular! ist 
die am einfachsten zu verwendende Tabellen-Umgebung von \LaTeX\ und gleichzeitig 
auch diejenige mit den wenigsten Layout-Möglichkeiten. 

\fbox{\texttt{\textbackslash
begin\{tabular\}[}\textsl{Position}\texttt{]\{}\textsl{Spalten}\texttt{\}\textsl{Tabelleninhalt}\textbackslash end\{tabular\}}}



In der ersten Zeile der Tabellendefinition,\index{Tabellendefinition} 
der sogenannten Tabellenpräambel\index{Tabellenpräambel} 
(das ist die Zeile mit 
dem \verb!\begin{tabular}!) ist die Anzahl der Spalten (\textsl{Spalten}) 
und die Ausrichtung der Tabelle (\textsl{Position}) 
definiert. Die Ausrichtung ist allerdings ein optionaler Parameter.

der Parameter \textsl{Spalten} definiert nicht nur die
Anzahl der Spalten\index{Spalten}, sondern auch die 
Ausrichtung des Inhalts in den Spalten (siehe 
Tabelle~\ref{Tabelle_Spaltenformatierungseintrag}).
Hier ist festgelegt, ob der Inhalt in den Zellen 
der Tabelle linksbündig, rechtsbündig oder 
zentriert ausgerichtet ist. Zudem wird innerhalb dieses Parameters
definiert, ob und wie viele vertikale Linien es in der Tabelle gibt,
um die einzelnen Spalten voneinander abzugrenzen.



Der Parameter \textsl{Position} 
kann die Werte \verb!t! oder \verb!b!
enthalten und definiert die vertikale 
Ausrichtung der Tabelle. 
Beim Wert 
\verb!t! wird die oberste Zeile
der Tabelle an der laufenden Umgebung 
ausgerichtet. Beim Wert \verb!b!
wird die Tabelle 
mit der untersten Zeile an der
laufenden Umgebung ausrichtet. 
Wird auf den Parameter \textsl{Position} 
verzichtet, dann wird die Tabelle in ihrer
vertikalen Mitte auf die laufende 
Umgebung ausgerichtet. 

Der Inhalt des Parameters \textsl{Spalten} hängt von 
folgenden Fragestellungen ab:

\begin{enumerate}
\item Wie viele Spalten enthält die Tabelle?
\item Wie soll der Inhalt in jeder Spalte ausgerichtet sein (links-,
rechtsbündig oder zentriert)?
\item Sollen die Spalten durch vertikale Linien kenntlich gemacht sein, und
wenn ja, welche Spalten und mit wie vielen Linien (eine oder zwei Linien)?
\end{enumerate}


% Jede Spalte der Tabelle müssen Sie im Parameter \textsl{spalten} angeben, und
% das geht so: 

Für jede Spalte, deren Inhalt 
linksbündig ausgerichtet sein soll,
enthält der Parameter \textsl{Spalten} ein mal den Buchstaben \verb!l!. 
Für jede Spalte, deren Inhalt rechtsbündig
ausgerichtet sein soll, wird ein \verb!r! eingefügt, und für eine Spalte
mit zentriertem Inhalt 
enthält der Parameter den Buchstaben \verb!c!.

Es ist auch möglich, eine linksbündig formatierte Spalte mit einer 
definierten Breite zu erstellen. Dieses geschieht mit einem Eintrag
\verb!p{!\textsl{Breite}\verb!}!. Eine Besonderheit dieses 
Spaltenformatierungseintrags ist, dass er die Realisierung mehrzeiliger
Spalteneinträge ermöglicht. Übersteigt der Inhalt eines Feldes die  
definierte Spaltenbreite, wird dieser in mehrere Zeilen umbrochen.
Mit den Spaltenformatierungseinträgen \verb!l!, \verb!r! und \verb!c! 
sind nur einzeilige Felder möglich.

Ein weiterer Spaltenformatierungseintrag,\index{Spaltenformatierungseintrag}
ist \verb!*{!\textsl{Anzahl}\verb!}{!\textsl{Spaltenform}\verb!}!.
Von der Stelle des Aufrufs im Parameter \textsl{Spalten}
wird die in \textsl{Spaltenform} festgelegte Spaltenformatierung
\textsl{Anzahl}-mal wiederholt.

Soll ein bestimmter Text oder ein sonstiger Inhalt in jeder 
Zeile der Tabelle zwischen zwei bestimmten Spalten erscheinen, kann dieses
mit dem Spaltenformatierungseintrag \texttt{!\{}\textsl{Text}\texttt{\}}
im Parameter \textsl{Spalten} angewiesen werden.

\begin{table}[htb]
	\centering
	\caption{Mögliche Einträge im Parameter \textsl{Spalten}}
	\label{Tabelle_Spaltenformatierungseintrag}       % Give a unique label
	\begin{tabularx}{\textwidth}{lX}
		\hline
		Wert & Bedeutung \\
		\hline
		\texttt{l} & Der Inhalt der Spalte wird linksbündig gesetzt. \\
		\texttt{r} & Der Inhalt der Spalte wird rechtsbündig gesetzt. \\
		\texttt{c} & Der Inhalt der Spalte wird zentriert gesetzt. \\
		\texttt{p\{}\textsl{Breite}\texttt{\}} & Die Spalte wird mit der Breite \texttt{Breite} gesetzt und der Inhalt wird, wenn er die definierte Spaltenbreite übersteigt, in mehrere Zeilen gebrochen.  \\
		\texttt{*\{}\textsl{Anzahl}\texttt{\}\{}\textsl{Spaltenform}\texttt{\}} & 
		Die in \textsl{Spaltenform} stehende Spaltenformatierung 
		wird \textsl{Anzahl}-mal wiederholt. \\
		\texttt{\textbar} & Es wird ein senkrechter Strich gesetzt. \\
		\texttt{\textbar\textbar} & Es wird ein senkrechter Doppelstrich gesetzt. \\
		\texttt{!\{}\textsl{Text}\texttt{\}} & Der \textsl{Text} wird in jeder Zeile
		zwischen die beiden 
		Spalten links und rechts davon gesetzt.  \\
		\hline
	\end{tabularx}
	\index{Linksbündige Spalte}
	\index{Rechtsbündige Spalte}
	\index{Zentrierte Spalte}
	\index{Tabellen!Strich}
	\index{Tabellen!Doppelstrich}
\end{table}


Senkrechte Striche oder Doppelstriche zum Kennzeichnen der 
Grenzen von Tabellen und/oder Spalten erzeugt man mit 
Hilfe von einzelnen oder doppelten 
senkrechten Strichen im Parameter \textsl{Spalten} 
(siehe Tabelle~\ref{Tabelle_Spaltenformatierungseintrag}).\index{Tabellen!Spalten}




Der Befehl \verb!\\! markiert das Ende jeder Zeile in einer Tabelle.
Zwischen zwei Spalten befindet sich immer das Zeichen \verb!&!. Auf diese
Art und Weise sind die einzelnen Zellen der Tabellen voneinander abgegrenzt.


\begin{table}[htb]
	\centering
	\caption{Eine einfache Tabelle}
	\label{Tabelle_Spaltenformatierungseintrag1}
	\begin{tabular}{lcr}
		\textbf{Linksbündig} & \textbf{Zentriert} & \textbf{Rechtsbündig} \\
		x1y1 & x2y1 & x3y1 \\
		x1y2 & x2y2 & x3y2 \\
		x1y3 & x2y3 & x3y3 \\
	\end{tabular}
\end{table}

Tabelle~\ref{Tabelle_Spaltenformatierungseintrag1} ist ein einfaches Beispiel 
für eine Tabelle mit drei Spalten. Der Quelltext von Tabelle~\ref{Tabelle_Spaltenformatierungseintrag1} befindet sich in Listing~\ref{erstestabellenbeispiel}. In der Spaltendeklaration der Tabellenpräambel
ist definiert, das die erste Spalte linksbündig (\verb!l!), die zweite Spalte zentriert (\verb!c!) und die vierte Spalte rechtsbündig (\verb!r!) ist.

\begin{lstlisting}[caption={Eine einfache Tabelle},label=erstestabellenbeispiel, style=customlatex]
\begin{tabular}{lcr}
\textbf{Linksbündig} & \textbf{Zentriert} & \textbf{Rechtsbündig} \\
x1y1 & x2y1 & x3y1 \\
x1y2 & x2y2 & x3y2 \\
x1y3 & x2y3 & x3y3 \\
\end{tabular}
\end{lstlisting}

Zur besseren Übersichtlichkeit ist es in vielen Publikationen üblich die einzelnen Zellen in Tabellen durch horizontale und vertikale Linien voneinander abzugrenzen. 
Tabelle~\ref{Tabelle_Spaltenformatierungseintrag2} erweitert das Beispiel aus Tabelle~\ref{Tabelle_Spaltenformatierungseintrag1} um horizontale und vertikale Linien. Im Beispiel sind alle Zellen der Tabelle durch vertikale und horizontale Linien voneinander abgegrenzt.
Der Quelltext von Tabelle~\ref{Tabelle_Spaltenformatierungseintrag2} befindet sich in Listing~\ref{zweitestabellenbeispiel}. 


\begin{table}[h!tb]
\centering
\caption{Vertikale und horizontale Linien trennen die Zellen voneinander ab}
\label{Tabelle_Spaltenformatierungseintrag2}
\begin{tabular}{|l|c|r|}
\hline
\textbf{Linksbündig} & \textbf{Zentriert} & \textbf{Rechtsbündig} \\
\hline\hline
x1y1 & x2y1 & x3y1 \\
x1y2 & x2y2 & x3y2 \\
x1y3 & x2y3 & x3y3 \\
\hline
\end{tabular}
\end{table}



\begin{lstlisting}[caption={Vertikale und horizontale Linien grenzen die Zellen voneinander ab},label=zweitestabellenbeispiel, style=customlatex]
\begin{tabular}{|l|c|r|}
\hline
\textbf{Linksbündig} & \textbf{Zentriert} & \textbf{Rechtsbündig} \\
\hline\hline
x1y1 & x2y1 & x3y1 \\
x1y2 & x2y2 & x3y2 \\
x1y3 & x2y3 & x3y3 \\
\hline
\end{tabular}
\end{lstlisting}

In der Spaltendeklaration in Listing~\ref{zweitestabellenbeispiel} sind die vertikalen Linien definiert.
Horizontalen Linien werden mit Hilfe des Befehls \verb!\hline! erzeugt.

Optisch eleganter ist es in vielen Fällen, auf vertikale Trennlinien zu verzichten und
nur zur Begrenzung der eigentlichen Tabelle, sowie des Tabellenkopfs, horizontale Trennlinien zu ziehen. 
Ein Beispiel für eine solche Tabelle ist Tabelle~\ref{Tabelle_Spaltenformatierungseintrag3}. Der zugehörige Quelltext ist Listing~\ref{drittetabellenbeispiel}. 



\begin{table}[h!tb]
\centering
\caption{Ohne vertikale Trennlinien wird die Tabelle optisch \emph{leichter}}
\label{Tabelle_Spaltenformatierungseintrag3}
\begin{tabular}{lcr}
\hline
\textbf{Linksbündig} & \textbf{Zentriert} & \textbf{Rechtsbündig} \\
\hline
x1y1 & x2y1 & x3y1 \\
x1y2 & x2y2 & x3y2 \\
x1y3 & x2y3 & x3y3 \\
\hline
\end{tabular}
\end{table}



\begin{lstlisting}[caption={Ohne vertikale Trennlinien wird die Tabelle optisch \emph{leichter}},label=drittetabellenbeispiel, style=customlatex]
\begin{tabular}{lcr}
\hline
\textbf{Linksbündig} & \textbf{Zentriert} & \textbf{Rechtsbündig} \\
\hline
x1y1 & x2y1 & x3y1 \\
x1y2 & x2y2 & x3y2 \\
x1y3 & x2y3 & x3y3 \\
\hline
\end{tabular}
\end{lstlisting}

Je nach gewünschter Struktur der Tabelle, ist es möglich, den Inhalt der Spaltendeklaration mit 
\verb!*{!\textsl{Anzahl}\verb!}{!\textsl{Spaltenform}\verb!}! zu vereinfachen.

Damit wird die in \textsl{Spaltenform} definierte
Spaltenform \textsl{Anzahl}-mal wiederholt. Enthält eine Tabelle beispielsweise in der Spaltendeklaration die Sequenz \verb!|l|l|l|l|!, kann diese durch \verb!*{4}{|l}|! verkürzt werden. Ein weiteres Beispiel ist die Sequenz \verb!|l|r|l|r|l|r|!. Diese kann mit Hilfe des Eintrags \verb!*{3}{|l|r}|! verkürzt werden.


Soll in jeder Zeile zwischen zwei bestimmten
Spalten der gleiche Inhalt eingefügt werden, beispielsweise ein Gleichheitszeichen (\verb!=!) oder ein Pfeil (z.B. \(\Longrightarrow\)), kann dieses einfach mit dem Spaltenformatierungseintrag \texttt{!\{}\textsl{Text}\texttt{\}} angewiesen werden. Ein sinnvolles Beispiel dieses Spaltenformatierungseintrags zeigt Tabelle~\ref{Tabelle_Spaltenformatierungseintrag4} mit einer Übersicht der binomische Formeln. Der zugehörige Quelltext ist Listing~\ref{viertestabellenbeispiel}. 

\begin{table}[h!tb]
\centering
\caption{Identischen Inhalt in jeder Zeile zwischen zwei Spalten einfügen}
\label{Tabelle_Spaltenformatierungseintrag4}
\begin{tabular}{r!{=}l}
\hline
\((a+b)^{2}\)     & \(a^{2}+2ab+b^{2}\) \\
\((a-b)^{2}\)     & \(a^{2}-2ab+b^{2}\) \\
\((a+b) * (a-b)\) & \(a^{2}-b^{2}\)     \\
\hline
\end{tabular}
\end{table}






\begin{lstlisting}[caption={Identischen Inhalt in jeder Zeile zwischen zwei Spalten einfügen},label=viertestabellenbeispiel, style=customlatex]
\begin{tabular}{r!{=}l}
\hline
\((a+b)^{2}\)     & \(a^{2}+2ab+b^{2}\) \\
\((a-b)^{2}\)     & \(a^{2}-2ab+b^{2}\) \\
\((a+b) * (a-b)\) & \(a^{2}-b^{2}\)     \\
\hline
\end{tabular}
\end{lstlisting}





\section{Die Umgebung \texttt{tabularx}}
\index[cmd]{\texttt{tabularx}}

Außer der Umgebung \verb!tabular! bietet \LaTeX\ zahlreiche weitere Umgebungen, um Tabellen zu setzen, die je nach konkretem Anwendungsfall hilfreich sein können. Eine dieser Umgebungen ist die Umgebung \verb!tabularx!, denn diese ermöglicht es Tabellen mit definierbarer Breite und automatischer Spaltenbreite zu realisieren.


\fbox{\texttt{\textbackslash
begin\{tabularx\}\{}\textsl{Breite}\texttt{\}[}\textsl{Position}\texttt{]\{}\textsl{Spalten}\texttt{\} 
Zeilen \textbackslash end\{tabularx\}}}

Um diese Umgebung zu verwenden, muss die Präambel der \verb!.tex!-Quelldatei den Befehl 
\verb!\usepackage{tabularx}! enthalten, um das Erweiterungspaket \verb!tabularx! einzubinden.

Das Erweiterungspaket definiert u.a. die Spaltenspezifikation \verb!X!. Diese realisiert eine Spalte mit linksbündigem Inhalt, deren Breite der \LaTeX-Compiler automatisch festlegt. Das Beispiel in Tabelle~\ref{Tabelle_tabularx1} zeigt die Funktionsweise. Jede Spalte erhält den gleichen Anteil an der verfügbaren Tabellenbreite. Der zugehörige Quelltext ist Listing~\ref{tabularx1beispiel}. 


\begin{table}[h!tb]
\centering
\caption{Jede Spalte erhält den gleichen Anteil an der verfügbaren Tabellenbreite}
\label{Tabelle_tabularx1}
\begin{tabularx}{\textwidth}{|X|X|X|}
\hline
Spalte 1 & Spalte 2 & Spalte 3 \\
\hline\hline
x1y1 & x2y1 & x3y1 \\
x1y2 & x2y2 & x3y2 \\
x1y3 & x2y3 & x3y3 \\
\hline
\end{tabularx}
\end{table}



\begin{lstlisting}[caption={Eine einfache Tabelle mit der Umgebung \texttt{tabularx}},label=tabularx1beispiel, style=customlatex]
\begin{tabularx}{\textwidth}{|X|X|X|}
\hline
Spalte 1 & Spalte 2 & Spalte 3 \\
\hline\hline
x1y1 & x2y1 & x3y1 \\
x1y2 & x2y2 & x3y2 \\
x1y3 & x2y3 & x3y3 \\
\hline
\end{tabularx}
\end{lstlisting}


Wie das Beispiel in Tabelle~\ref{Tabelle_tabularx1} zeigt, 
sorgt die Spaltenspezifikation \verb!X! 
dafür, dass jede Spalte den gleichen Anteil an der verfügbaren
Tabellenbreite bekommt. Die Berechnung der Spaltenbreite erfolgt anhand folgender 
Formel:

\begin{displaymath}
Spaltenbreite = \frac{Gesamtbreite\ der\ Tabelle}{Spaltenanzahl}
\end{displaymath}


Im Beispiel ist die Tabellenbreite identisch mit der
Breite des Textfelds im Dokument. Alternativ könnte beispielsweise 
ein feste Breite unter Angabe eines Werts inklusive einer Maßeinheit 
(siehe Abschnitt~\ref{sec:Massangaben}) definiert sein.

Standardmäßig ermöglicht \verb!tabularx! nur linksbündige Spalten mit automatischer Breite. Sollen auch Spalten mit rechtsbündigem oder zentriertem Inhalt mit automatischer Breite möglich sein, ermöglichen dieses die beiden folgenden Zeilen in der Präambel der \verb!.tex!-Quelldatei.


\fbox{\texttt{\textbackslash newcolumntype\{Y\}\{>\{\textbackslash centering\textbackslash arraybackslash\}X\}}}

\fbox{\texttt{\textbackslash newcolumntype\{Z\}\{>\{\textbackslash hfill\textbackslash arraybackslash\}X\}}}


Die beiden obigen Zeilen definieren die beiden Spaltenspezifikationen \verb!Y! für
Spalten mit zentriertem Inhalt und \verb!Z! für Spalten mit rechtsbündigem Inhalt. Das Beispiel in Tabelle~\ref{Tabelle_tabularx2} zeigt die praktische Anwendung der 
neu definierten Spaltenspezifikationen \verb!Y! und \verb!Z!. Der zugehörige Quelltext ist Listing~\ref{tabularx2beispiel}. 


\begin{table}[h!tb]
	\centering
	\caption{Spalten mit automatischer Breite bei \texttt{tabularx}}
	\label{Tabelle_tabularx2}
	\begin{tabularx}{10cm}{|X|Y|Z|}
		\hline
		Spalte 1 & Spalte 2 & Spalte 3 \\
		\hline\hline
		x1y1 & x2y1 & x3y1 \\
		x1y2 & x2y2 & x3y2 \\
		x1y3 & x2y3 & x3y3 \\
		\hline
	\end{tabularx}
\end{table}


Für Tabelle~\ref{Tabelle_tabularx2} ist eine Tabellenbreite von 10\,cm definiert. Die drei Spalten erhalten automatisch den gleichen Anteil an der verfügbaren
Tabellenbreite.



\begin{lstlisting}[caption={Spalten mit automatischer Breite bei \texttt{tabularx}},label=tabularx2beispiel, style=customlatex]
\newcolumntype{Y}{>{\centering\arraybackslash}X}
\newcolumntype{Z}{>{\hfill\arraybackslash}X}

...

\begin{tabularx}{.8\textwidth}{|X|Y|Z|}

Spalte 1 & Spalte 2 & Spalte 3 \\
\hline\hline
x1y1 & x2y1 & x3y1 \\
x1y2 & x2y2 & x3y2 \\
x1y3 & x2y3 & x3y3 \\
\hline
\end{tabularx}
\end{lstlisting}


Es ist auch möglich, in einer Tabelle, die mit \verb!tabularx! gesetzt wird, die Spaltenspezifikationen \verb!l!, \verb!r! und \verb!c! aus der Umgebung \verb!tabular! zu verwenden. In diesem Fall orientiert sich die Spaltenbreite wie gehabt am Inhalt. Enthält die Tabelle auch eine oder mehr Spalten mit automatischer Breite, werden diese entsprechend gestreckt oder gestaucht, damit die im Parameter \textsl{Breite} definierte Gesamtbreite der Tabelle eingehalten wird. Tabelle~\ref{Tabelle_tabularx3} und der zugehörige Quelltext in Listing~\ref{tabularx3beispiel} zeigen dieses Verhalten.



\begin{table}[h!tb]
\centering
\caption{Kombination der Spaltenspezifikationen von \texttt{tabular} und \texttt{tabularx}}
\label{Tabelle_tabularx3}
\begin{tabularx}{10cm}{|X|c|r|}
\hline
  Spalte 1 & Spalte 2 & Spalte 3 \\
\hline\hline
x1y1 & x2y1 & x3y1 \\
x1y2 & x2y2 & x3y2 \\
x1y3 & x2y3 & x3y3 \\
\hline
\end{tabularx}
\end{table}



\begin{lstlisting}[caption={Kombination der Spaltenspezifikationen von \texttt{tabular} und \texttt{tabularx}},label=tabularx3beispiel, style=customlatex]
\begin{tabularx}{10cm}{|X|c|r|}
\hline
  Spalte 1 & Spalte 2 & Spalte 3 \\
\hline\hline
x1y1 & x2y1 & x3y1 \\
x1y2 & x2y2 & x3y2 \\
x1y3 & x2y3 & x3y3 \\
\hline
\end{tabularx}
\end{lstlisting}


\section{Farbige Tabellen}
\index{Tabellen!Farben}

Zum Einfärben von Tabellen stehen nach dem Einbinden des Erweiterungspakets
\verb!colortbl.sty! zahlreiche Befehle zur Verfügung.  
Der Import des Erweiterungspakets geschieht mit dem Befehl 
\verb!\usepackage{colortbl}! in der Präambel der \verb!.tex!-Quelldatei. Zudem müssen noch die Erweiterungspakete \verb!xcolor! und \verb!array! in der Präambel eingebunden sein.

Der Befehl \verb!\rowcolor!\index[cmd]{\texttt{\textbackslash rowcolor}} 
ermöglicht das Einfärben einzelner Zeilen.

\fbox{\texttt{\textbackslash rowcolor[}\textsl{Farbmodell}\texttt{]\{}\textsl{Farbe}\texttt{\}}}

Das optionale Argument \verb!Farbmodell! definiert das gewünschte Farbmodell.
Mögliche Angaben sind \verb!cmyk!, \verb!HTML!, \verb!gray!, \verb!rgb! und \verb!RGB!.
Das zwingend erforderliche Argument \verb!Farbe! definiert eine Farbe innerhalb des gewählten Farbmodells. Eine Übersicht über die  unterstützten Farbmodelle enthält Tabelle~\ref{Tabelle_Farbmodelle}\index{Farbmodell}.


\begin{table}[h!tb]
\centering
\caption{Übersicht über die unterstützten Farbmodelle}
\label{Tabelle_Farbmodelle}       % Give a unique label
\begin{tabularx}{\textwidth}{lX}
\hline
Farbmodell & Bedeutung\\
\hline
\texttt{cmyk}\index{CMYK} & Bei diesem Farbmodell besteht jede Farbe aus vier Werten, stellvertretend für Cyan, Magenta, Gelb und Schwarz. Die vier Werte liegen im Zahlenraum zwischen 0 und 1 und sind durch Kommas voneinander abgetrennt. \\
\hline
\texttt{gray} & Die Farbangabe bei diesem Farbmodell besteht nur aus einem Wert zwischen
0 für schwarz und 1 für weiß.\\
\hline
\texttt{HTML}\index{HTML} & Bei diesem Farbmodell besteht jede Farbe aus hexadezimalen Werten in der Form RRGGBB. Für jede der drei Grundfarben wird ein Wert (00-FF) angegeben. Die Funktionsweise ist identisch mit Farbangaben bei der Auszeichnungssprache HTML. \\
\hline
\texttt{rgb} & Dieses Farbmodell steht für \textsl{Rot-Grün-Blau}. Hier besteht jede Farbangabe aus drei durch Kommas abgetrennten Werten zwischen 0 und 1. 
Der jeweilige Wert steht für den Anteil an rotem, grünem und blauem Licht.\\
\hline
\texttt{RGB}\index{RGB} & Auch dieses Farbmodell steht für \textsl{Rot-Grün-Blau}. Hier besteht jede Farbangabe aus drei durch Kommas abgetrennten Werten zwischen 0 und 255. \\
\hline
\end{tabularx}
\end{table}


Ein Beispiel zum Befehl \verb!\rowcolor! enthält Tabelle~\ref{Tabelle_Farbige_Zeilen1}. In dieser wurden einzelne Zeilen mit dem Farbmodell \verb!gray! eingefärbt. Der zugehörige Quelltext befindet sich in Listing~\ref{tabularfarbe1}.  


\begin{table}[htb]
\centering
\caption{Einfärben ganzer Zeilen einer Tabelle}
\label{Tabelle_Farbige_Zeilen1}
\begin{tabular}{lcr}
\hline
Spalte 1 & Spalte 2 & Spalte 3 \\
\hline\hline
\rowcolor[gray]{0.9} x1y1 & x2y1 & x3y1 \\
\rowcolor[gray]{0.8} x1y2 & x2y2 & x3y2 \\
\rowcolor[gray]{0.7} x1y3 & x2y3 & x3y3 \\
\rowcolor[gray]{0.6} x1y4 & x2y4 & x3y4 \\
\rowcolor[gray]{0.5} x1y5 & x2y5 & x3y5 \\
\rowcolor[gray]{0.4} x1y6 & x2y6 & x3y6 \\
\rowcolor[gray]{0.3} x1y7 & x2y7 & x3y7 \\
\rowcolor[gray]{0.2} x1y8 & x2y8 & x3y8 \\
\rowcolor[gray]{0.1} x1y9 & x2y9 & x3y9 \\
\hline
\end{tabular}
\end{table}





\begin{lstlisting}[caption={Das Einfärben ganzer Tabellenzeilen ermöglicht der Befehl \texttt{rowcolor}},label=tabularfarbe1, style=customlatex]
\begin{tabular}{lcr}
\hline
Spalte 1 & Spalte 2 & Spalte 3 \\
\hline\hline
\rowcolor[gray]{0.9} x1y1 & x2y1 & x3y1 \\
\rowcolor[gray]{0.8} x1y2 & x2y2 & x3y2 \\
\rowcolor[gray]{0.7} x1y3 & x2y3 & x3y3 \\
\rowcolor[gray]{0.6} x1y4 & x2y4 & x3y4 \\
\rowcolor[gray]{0.5} x1y5 & x2y5 & x3y5 \\
\rowcolor[gray]{0.4} x1y6 & x2y6 & x3y6 \\
\rowcolor[gray]{0.3} x1y7 & x2y7 & x3y7 \\
\rowcolor[gray]{0.2} x1y8 & x2y8 & x3y8 \\
\rowcolor[gray]{0.1} x1y9 & x2y9 & x3y9 \\
\hline
\end{tabular}
\end{lstlisting}


Das Einfärben von Spalten ermöglicht der Befehl \verb!\columncolor!\index[cmd]{\texttt{\textbackslash columncolor}}.


\fbox{\texttt{\textbackslash columncolor[}\textsl{Farbmodell}\texttt{]\{}\textsl{Farbe}\texttt{\}[}\textsl{UeberhangL}\texttt{][}\textsl{UeberhangR}\texttt{]}}

Die Bedeutung der beiden Argumente \textsl{Farbmodell} und \textsl{Farbe} ist bereits vom Befehl \verb!\rowcolor! bekannt. Die beiden optionalen Argumente
\textsl{UeberhangL} und \textsl{UeberhangR} definieren den Überhang auf der linken bzw. rechten Seite. Mit \textsl{UeberhangL} und \textsl{UeberhangR} können Autoren also festlegen, wie weit die Färbung links und 
rechts über den Inhalt der Spalte hinausragt. Da diese beiden Argumente in der Praxis eher selten zum Einsatz kommen, werden Sie in diesem Werk nicht weiter betrachtet.


Ein Beispiel zum Befehl \verb!\columncolor! enthält Tabelle~\ref{Tabelle_Farbige_Spalten1}. Der zugehörige Quelltext ist Listing~\ref{tabularfarbe2}. Auch in diesem Beispiel wurden einzelne Zeilen mit dem Farbmodell \verb!gray! eingefärbt. Der Befehl \verb!\columncolor! wird in der Spaltendeklaration vor jeder einzufärbenden Spalte aufgerufen.


\begin{table}[h!tb]
\centering
\caption{Einfärben ganzer Spalten einer Tabelle}
\label{Tabelle_Farbige_Spalten1}
\begin{tabular}{|>{\columncolor[gray]{0.9}}c
                |>{\columncolor[gray]{0.7}}c
                |>{\columncolor[gray]{0.5}}c
                |>{\columncolor[gray]{0.3}}c
                |>{\columncolor[gray]{0.1}}c|}
\hline 
Spalte 1 & Spalte 2 & Spalte 3 & Spalte 4 & Spalte 5  \\
\hline\hline
x1y1 & x2y1 & x3y1 & x4y1 & x5y1 \\
x1y2 & x2y2 & x3y2 & x4y2 & x5y2 \\
x1y3 & x2y3 & x3y3 & x4y3 & x5y3 \\
\hline
\end{tabular}
\end{table}



\begin{lstlisting}[caption={Das Einfärben ganzer Tabellenspalten ermöglicht der Befehl \texttt{columncolor}},label=tabularfarbe2, style=customlatex]
\begin{tabular}{|>{\columncolor[gray]{0.9}}c
                |>{\columncolor[gray]{0.7}}c
                |>{\columncolor[gray]{0.5}}c
                |>{\columncolor[gray]{0.3}}c
                |>{\columncolor[gray]{0.1}}c|}
\hline 
Spalte 1 & Spalte 2 & Spalte 3 & Spalte 4 & Spalte 5  \\
\hline\hline
x1y1 & x2y1 & x3y1 & x4y1 & x5y1 \\
x1y2 & x2y2 & x3y2 & x4y2 & x5y2 \\
x1y3 & x2y3 & x3y3 & x4y3 & x5y3 \\
\hline
\end{tabular}
\end{lstlisting}

Das Einfärben einzelner Felder einer Tabelle geschieht mit dem Befehl \verb!\cellcolor!\index[cmd]{\texttt{\textbackslash cellcolor}}.

\fbox{\texttt{\textbackslash cellcolor[}\textsl{Farbmodell}\texttt{]\{}\textsl{Farbe}\texttt{\}}}

Ein Beispiel zum Befehl \verb!\cellcolor! enthält Tabelle~\ref{Tabelle_Farbige_Zellen1}. Der zugehörige Quelltext ist Listing~\ref{tabularfarbe3}.

\begin{table}[h!tb]
\centering
\caption{Einfärben einzelner Felder einer Tabelle}
\label{Tabelle_Farbige_Zellen1}
\begin{tabular}{|c|c|c|}
\hline
Spalte 1 & Spalte 2 & Spalte 3 \\
\hline\hline
\cellcolor[gray]{0.9} x1y1 & 
\cellcolor[gray]{0.8} x2y1 & 
\cellcolor[gray]{0.7} x3y1 \\
\cellcolor[gray]{0.6} x1y2 & 
\cellcolor[gray]{0.5} x2y2 & 
\cellcolor[gray]{0.4} x3y2 \\
\cellcolor[gray]{0.3} x1y3 & 
\cellcolor[gray]{0.2} x2y3 & 
\cellcolor[gray]{0.1} x3y3 \\
\hline
\end{tabular}
\end{table}



\begin{lstlisting}[caption={Das Einfärben einzelner Tabellenfelder (Zellen) ermöglicht der Befehl \texttt{cellcolor}},label=tabularfarbe3, style=customlatex]
\begin{tabular}{|c|c|c|}
\hline
Spalte 1 & Spalte 2 & Spalte 3 \\
\hline\hline
\cellcolor[gray]{0.9} x1y1 & 
\cellcolor[gray]{0.8} x2y1 & 
\cellcolor[gray]{0.7} x3y1 \\
\cellcolor[gray]{0.6} x1y2 & 
\cellcolor[gray]{0.5} x2y2 & 
\cellcolor[gray]{0.4} x3y2 \\
\cellcolor[gray]{0.3} x1y3 & 
\cellcolor[gray]{0.2} x2y3 & 
\cellcolor[gray]{0.1} x3y3 \\
\hline
\end{tabular}
\end{lstlisting}

Wird eine Farbe häufiger im Dokument verwendet, ist es sinnvoll, diese Farbe mit dem 
Befehl \verb!\definecolor!\index[cmd]{\texttt{\textbackslash definecolor}} in der Präambel des Quelltextes zu definieren und auf diese definierte Farbe immer wieder zuzugreifen. Dieses Vorgehen vereinfacht den Quelltext und eventuelle Änderungen an der Farbauswahl müssen später nur an einer einzigen Stelle vorgenommen werden. Beispielhaft für jedes der in Tabelle~\ref{Tabelle_Farbmodelle} vorgestellten Farbmodelle definieren die folgenden Befehle die Farbe Deep Sky Blue.


\begin{Verbatim}[frame=single]
\definecolor{deepskyblue}{cmyk}{1,0.25,0,0}
\definecolor{deepskyblue}{HTML}{00BFFF}
\definecolor{deepskyblue}{rgb}{0.0,0.75,1.0}
\definecolor{deepskyblue}{RGB}{0,191,255}
\end{Verbatim}



Mit dem Farbmodell \verb!gray! kann die Farbe Deep Sky Blue natürlich nicht dargestellt werden. Darum definiert der folgende Befehl exemplarisch ein helles Grau.

\begin{Verbatim}[frame=single]
\definecolor{lightgray}{gray}{0.6}
\end{Verbatim}

Anschließend ist via \verb!\rowcolor{deepskyblue}!, \verb!\columncolor{deepskyblue}! oder \verb!\cellcolor{deepskyblue}! die definierte Farbe nutzbar.



\section{Felder zusammenfassen}
\index{Tabellen!Felder zusammenfassen}

Das Zusammenfassen von Feldern über die Grenzen von Spalten hinweg, geschieht mit dem Befehl \verb!\multicolumn!\index[cmd]{\texttt{\textbackslash multicolumn}}. 

\fbox{\texttt{\textbackslash multicolumn\{}\textsl{Anzahl}\texttt{\}\{}\textsl{Format}\texttt{\}\{}\textsl{Inhalt}\texttt{\}}}

Um diesen Befehl zu verwenden, muss das gleichnamige Erweiterungspaket mit dem Befehl \verb!\usepackage{multicol}! in der Präambel der \verb!.tex!-Quelldatei eingebunden sein.

Der Parameter \textsl{Anzahl} definiert wie viele Felder zusammengefasst werden sollen und im Parameter \textsl{Format} wird mit den bekannten Positionszeichen \verb!l!, \verb!r! oder \verb!c! gibt an, ob der Inhalt des neuen Felds linksbündig, rechtsbündig oder zentriert gesetzt wird. Auch vertikale Linien können hier mit \verb!|! oder \verb!||! angewiesen werden.

Ein Beispiel zum Befehl \verb!\multicolumn! enthält Tabelle~\ref{Tabelle_Multicolumn1} und der zugehörige Quelltext befindet sich in Listing~\ref{tabularmulticolumn1}.


\begin{table}[h!tb]
\centering
\caption{Spaltenübergreifendes Zusammenfassen von Feldern einer Tabelle}
\label{Tabelle_Multicolumn1}
\begin{tabular}{|c|c|c|}
\hline
\multicolumn{3}{|c|}{Oberste Zeile} \\
\hline\hline
eins & zwei & drei \\
\hline
\multicolumn{2}{|c|}{vier} & fünf \\
\hline
sechs & \multicolumn{2}{c|}{sieben} \\
\hline
\end{tabular}
\end{table}


\begin{lstlisting}[caption={Der Befehl \texttt{multicolumn} kann Felder über Spalten hinweg zusammenfassen},label=tabularmulticolumn1, style=customlatex]
\begin{tabular}{|c|c|c|}
\hline
\multicolumn{3}{|c|}{Oberste Zeile} \\
\hline\hline
eins & zwei & drei \\
\hline
\multicolumn{2}{|l|}{vier} & fünf \\
\hline
sechs & \multicolumn{2}{r|}{sieben} \\
\hline
\end{tabular}
\end{lstlisting}

Hilfreich ist der Befehl \verb!\multicolumn! auch dann, wenn nur die Ausrichtung des Inhalts eines Felds in einer Tabelle geändert werden soll. In
diesem Fall erhält der Parameter \textsl{Anzahl} den Wert \verb!1! und der Parameter \textsl{Format} das gewünschte Positionszeichen. 
Ein Beispiel zu diesem Vorgehen enthält Tabelle~\ref{Tabelle_Multicolumn2}. Der zugehörige Quelltext ist Listing~\ref{tabularmulticolumn2}.


\begin{table}[h!tb]
\centering
\caption{Änderung der Ausrichtung des Inhalts in einem einzelnen Feld}
\label{Tabelle_Multicolumn2}
\begin{tabular}{|p{2cm}|p{2cm}|p{2cm}|}
\hline
\multicolumn{3}{|c|}{Hier ist die oberste Zeile} \\
\hline\hline
links & links & links \\
\hline
links & \multicolumn{1}{r|}{\cellcolor[gray]{0.7} rechts} & links \\
\hline
links & links & links \\
\hline
\end{tabular}
\end{table}

\begin{lstlisting}[caption={Mit dem Befehl \texttt{multicolumn} kann auch die Ausrichtung nur eines einzelnen Feldes angepasst werden},label=tabularmulticolumn2, style=customlatex]
\begin{tabular}{|p{2cm}|p{2cm}|p{2cm}|}
\hline
\multicolumn{3}{|c|}{Hier ist die oberste Zeile} \\
\hline\hline
links & links & links \\
\hline
links & \multicolumn{1}{r|}{\cellcolor[gray]{0.7} rechts} & links \\
\hline
links & links & links \\
\hline
\end{tabular}
\end{lstlisting}


Das Zusammenfassen von Feldern über die Grenzen von Zeilen hinweg, geschieht mit dem Befehl \verb!multirow!.

\fbox{\texttt{\textbackslash multirow\{}\textsl{Anzahl}\texttt{\}\{}\textsl{Breite}\texttt{\}\{}\textsl{Inhalt}\texttt{\}}}

Um diesen Befehl zu verwenden, muss das gleichnamige Erweiterungspaket mit dem Befehl \verb!\usepackage{multirow}! in der Präambel der \verb!.tex!-Quelldatei eingebunden sein.

Der Parameter \textsl{Anzahl} definiert wie viele Felder zusammenzufassen sind und \textsl{Breite} legt die Breite des zusammengefassten Felds fest. Hat dieser Parameter den Wert \verb!*!, legt \LaTeX\ die Breite automatisch anhand des Inhalts fest.


Ein Beispiel zum Befehl \verb!\multirow!\index[cmd]{\texttt{\textbackslash multirow}}
enthält Tabelle~\ref{Tabelle_multirow1} und der zugehörige Quelltext befindet sich in Listing~\ref{tabularmmultirow1}.




\begin{table}[h!tb]
\centering
 \caption{Das zeilenübergreifende Zusammenfassen von Feldern einer Tabelle ermöglicht der Befehl \texttt{multirow}}
\label{Tabelle_multirow1}
\begin{tabular}{|c|c|c|}
\hline
Spalte 1 & Spalte 2 & Spalte 3 \\
\hline\hline
\multirow{2}{*}{eins} & zwei & drei \\
                      & vier & fünf \\
\hline
\end{tabular}
\end{table}


\begin{lstlisting}[caption={Mit dem Befehl \texttt{multirow} können Felder zeilenübergreifend zusammengefasst werden}, label=tabularmmultirow1, style=customlatex]
\begin{tabular}{|c|c|c|}
\hline
Spalte 1 & Spalte 2 & Spalte 3 \\
\hline\hline
\multirow{2}{*}{eins} & zwei & drei \\
                      & vier & fünf \\
\hline
\end{tabular}
\end{lstlisting}


\section{Horizontale Linien}

Der Befehl \verb!\hline!\index[cmd]{\texttt{\textbackslash hline}}, mit dem einzelne oder doppelte horizontale Linien in Tabellen eingefügt werden, ist aus den Beispielen dieses Kapitel schon gut bekannt. Mit diesem Befehl ist es aber ausschließlich möglich, Linien über die komplette horizontale Breite einer Tabelle zu realisieren. 

Mehr Flexibilität bietet der Befehl \verb!\cline!\index[cmd]{\texttt{\textbackslash cline}}, der in der nach seinem Aufruf folgenden Zeile eine horizontale Linie einfügt, die sich vom
linken Rand der Spalte \textsl{SpalteL} bis zum rechten Rand der 
Spalte \textsl{SpalteR} erstreckt.

\fbox{\texttt{\textbackslash cline\{}\textsl{SpalteL -- SpalteR}\texttt{\}}}

Der mehrfache Aufruf von \verb!cline! direkt hintereinander ist zulässig. 
Soll beispielsweise unterhalb der gerade abgeschlossenen Zeile vom linken 
Rand der Spalte 1 bis zum rechten Rand der Spalte 3 und vom linken 
Rand der Spalte 5 bis zum rechten Rand der Spalte 9 eine 
horizontale Linie gesetzt werden, realisiert dieses die Befehlsfolge 
\verb!\cline{1-3}\cline{5-9}!.



Ein Beispiel wo es sinnvoll sein kann, horizontale Linien nur durch einen Teil einer Tabelle laufen zu lassen, ist wenn Felder mit dem Befehl 
\verb!\multirow!\index[cmd]{\texttt{\textbackslash multirow}} 
zusammengefasst werden. Ein Beispiel dafür zeigt Tabelle~\ref{Tabelle_Multirow2}. Der zugehörige Quelltext befindet sich in Listing~\ref{tabularcline}. Wäre in diesem Beispiel die Linie von der zweiten bis zur dritten Spalte anstatt mit dem Befehl \verb|\cline{2-3}| mit \verb|\hline| realisiert worden, hätte die horizontale Linie das zusammengefasste Feld in der ersten Spalte durchschnitten.







\begin{table}[h!tb]
\centering
\caption{Horizontale Linien mit fast beliebiger Breite in Tabellen ermöglicht der Befehl \texttt{cline}}
\label{Tabelle_Multirow2}
\begin{tabular}{|c|c|c|}
\hline
Spalte 1 & Spalte 2 & Spalte 3 \\
\hline\hline
\multirow{2}{*}{eins} & zwei & drei \\
\cline{2-3}
                      & vier & fünf \\
\hline
\end{tabular}
\end{table}

\begin{lstlisting}[caption={Horizontale Linien mit fast beliebiger Breite in Tabellen setzen},label=tabularcline, style=customlatex]
\begin{tabular}{|c|c|c|}
\hline
Spalte 1 & Spalte 2 & Spalte 3 \\
\hline\hline
\multirow{2}{*}{eins} & zwei & drei \\
\cline{2-3}
                      & vier & fünf \\
\hline
\end{tabular}
\end{lstlisting}
